\chapter{Differential Geometry, Lecture 1}

\begin{lemma}\label{DotProductDifferentiation}

\end{lemma}

\begin{lemma}\label{EvolventDistanceFormula}
    The distance between two points on the evolvent is given by
    \[
        \d (c(t_1), c(t_2) ) = \frac{ \lvert \kappa_\gamma(t_2) - \kappa_\gamma(t_1) \rvert}{\lvert \kappa_\gamma(t_1) \kappa_\gamma(t_2) \rvert } = \lvert r_1
    \]
\end{lemma}

\paragraph{Observation.} Curvature is the determinant of the Frenet frame.
\paragraph{Observation.} Walking the curve backwards negates the curvatues, but does not change its derivative!

\chapter{Differential Geometry, Problemset 1}

\section{Problem 5}

By \ref{EvolventDistanceFormula}, 

\section{Problem 8}

\paragraph{Geometry.} The curve \emph{rolls in on itself}.
% TODO: add picture of curve, which has increasing curvature whose range is the whole real line
The main idea is the below lemma, from which the problem follows immediately.
\begin{lemma}\label{IncreasingCurvatureInsideCircle}
    If a curve
    \[ \gamma: (a, b) \to \mathbb{R}^2 \]
    has increasing curvature, then for all \( t \), 
\end{lemma}
