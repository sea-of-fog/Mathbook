\chapter{Abelian groups}

We will now explore some features of abelian groups. These will later be
recontextualized into features of modules (for those of you know, an abelian
group is essentialy the same as a \mathbb{Z}-module). In this chapter, we will
derive a structure theorem.

\section{Basic properties}

% TODO: additive notation, statements about it

\section{Torsion}

Central to the study of abelian groups is the notion of torsion. We begin with~
two definitions, one for elements and one for groups.

\begin{defn}
    An element of an abelian group is called a \textbf{torsion element} if it is
    of finite order. An element of infinite order is called \textbf{torsion-free}.
\end{defn}

\begin{defn}
    An abelian group is called a \textbf{torsion group} if all its elements are torsion.
    An abelian gropu is called \textbf{torsion-free} if all its \emph{nonzero} elements
    are torsion-free.
\end{defn}

One might ask why we define this only for abelian groups. The following crucial fact
is the reason -- more precisely, the fact that its proof relies on commutativity.

\begin{thm}
    If \( \alpha, \beta \) are two torsion elements then their sum (and difference)
    is also a torsion element. Thus, the torsion elements form a subgroup.
\end{thm}

\begin{Proof}
Let \( \alpha \) have order \( n \) and \( \beta \) have order \( m \). Then we have
\[ 
    nm(\alpha + \beta) = nm\alpha + nm\beta =  m(n\alpha) + n(m\beta) = m \cdot 0 + n \cdot 0 = 0.
\]
The subgroup claim is proved by noting that the same computation hold as well
for difference and that the identity element is of course torsion.
\end{Proof}

% TODO: example of nonabelian group where the torsion elements are not subgroups

Since we have a well defined kind of subgroup, we might as well give it a name.

\begin{defn}
    Let \( G \) be an abelian group. Its \textbf{torsion subgroup}, denoted
    \[ 
        {\rm T}(G) {\rm or} Tor(G) 
   \]
   is the subgroup consisting of all the torsion elements of \( G \).
\end{defn}

We may hope that such a nice notion of substructure is also valid for the torsion-free
elements of a group. This fails, however, as the set of torsionfree elements is
the~complement of a subgroup, which will very often fail to be a subgroup. On the level
of elements, we have the following fact.

% TODO: correct the Tor to an operator
\begin{thm}
    Let \( a \in G \) be torsionfree and \( b \in Tor(G) \). Then \( a + b \) is torsionfree.
\end{thm}

\begin{proof}
    We give two proofs. For the first one, suppose \( a + b \) is torsion. Then
    \[ 
       a = (a + b) - b 
   \]
   would be torsion (since the torsion elements form a subgroup).

   For the second proof, suppose \( a + b \) is torsion with order \( n \) and
   that \( b \) has order \( m \). Then
   \[ 
      0 = mn(a + b) = mna + mnb = mna, 
   \]
   so \( a \) is torsion as well.
\end{proof}

All is not lost though! We cannot define a notion of \textit{torsion-free subgroup},
but there is a simple way of killing the torsion -- quotienting!

\subsection{A weird example of groups}
% TODO: finitely generated torsion-free group is free

\subsection{Functors}

\subsection{Tensors}
% TODO: torsion and tensors


\section{Structure theorem}

\section{A smudge of infinite abelian group theory: the prufer groups}

Prufer groups are: divisible, \( p \)-torsion, subgroups, infinitely generated

\section{Sources for this chapter}

\begin{enumerate}
    \item Ludomir Newelski, Algebra II
    \item Ludomir Newelski, Algebra 2R
    \item Wikipedia: torsion subgroup, torsion, torsion-free group, Prufer groups
\end{enumerate}

