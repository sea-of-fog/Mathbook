\chapter{Differential Geometry, Lecture 2}

\section{Curves in \( \mathbb{R}^3 \)}

Let \( \gamma: (a,b) \to \mathbb{R}^3 \) be parametrised by arclength and \( \gamma'' \neq 0 \). Then
\[ 
    \gamma' \perp \gamma'' 
\]
and we can define
\[ 
    N := \frac{ \gamma'' }{ \left \lvert\lvert \gamma'' \right\rvert\rvert  }.
\]
We need a third vector, ideally so that the three vectors are positively oriented. No worries, we can put down
\[ 
    B := \tau \times N. 
\]

\begin{defn}
    We define the curvature in \( \mathbb{R}^3 \) to be
    \[ 
       \kappa_\gamma := \left \lvert\lvert \gamma'' \right\rvert\rvert  
   \].
\end{defn}

\begin{defn}
    We define \emph{trójnóg Freneta} to be the ordered orthonormal basis
    \[ 
        (\tau, N, B).
   \]
\end{defn}

%TODO: name of the B vector
We have an analogue of the Frenet equations
\begin{thm}[3D Frenet Equations]
\label{FrenetEquations3D}
\hypertarget{FrenetEquations3D}
    For the \emph{Frenet trójnóg} we have
    \[ 
       \frac{\mathrm{d}^{}}{\mathrm{d}^{}t} (T, N, B) = (T,N,B) M
       %TODO
   \]
\end{thm}

\begin{proof}
The first row follows from the definition. Let
\[ 
    A = (T,N,B). 
\]
This is an orthonormal matrix for all \( t \in (a,b) \), so it traces a curve in \( \SO(3) \) and we want its derivative. We will show
\[ 
    A^{-1}  A'
\]
is skew-symmetric -- this will give us the theorem. We know that
\[ 
   AA^T = \Id,
\]
so let us differentiate both sides and obtain
\[ 
    (A')^TA + A^TA' = 0
\]
and
\[ 
    (A^TA')^T = (A')^TA = -A^TA'. 
\]
Where we have used the matrix Leibniz rule.
%TODO: matrix Leibniz rule
\end{proof}

\begin{defn}
    The function \( \tau \) mentioned in the previous theorem is called \textbf{torsion}.
\end{defn}

We have another analogue to curves in \( \mathbb{R}^2 \).

\begin{thm}[Fundamental Theorem of Curves is \( \mathbb{R}^3 \)]
\label{FundamentalCurveTheorem3D}
\hypertarget{FundamentalCurveTheorem3D}
For a positive function
\[ 
    \kappa: (a,b) \to \mathbb{R}_+,
\]
an arbitrary function
\[ 
    \tau: (a,b) \to \mathbb{R} 
\]
and any positively oriented orthonormal basis of \( \mathbb{R}^3 \) (the initial Frenet frame), there exists exactly one arclength parametrised curve
\[ 
    \gamma: (a,b) \to \mathbb{R}^3
\]
which has Frenet frame at \( t = 0 \) equal to \( B \).
\end{thm}

\begin{proof}[Proof (sketch).]
As in the 2-dimensional case -- put down a differential equation.
\end{proof}

\paragraph{What is torsion?} The torsion describes how much a curve escapes a plane in which it starts (the \emph{osculating plane}).

\section{Curvature of surfaces}

% TODO: inclusion arrow
Consider a surface 
\[ 
    S \subseteq \mathbb{R}^3.
\]
We will want to define its curvature in terms of plane curvature. We will work with surfaces given as immersions
\[ 
    \Sigma: U \to \mathbb{R}^3,
\]
where \( U \subseteq \mathbb{R}^2 \).

\begin{defn}
    Let \( p \in U \subseteq \mathbb{R}^2 \) and \( \Sigma \) be as above. The \textbf{Riemann metric} on \( TU \) is the smooth, positive definite bilinear form given at \( p \) by
    \[ 
        \left\langle v_p, w_p \right\rangle_\Sigma = \left\langle \mathrm{d}\Sigma(v_p), \mathrm{d}\Sigma(w_p) \right\rangle. 
   \]
\end{defn}

\paragraph{Notation.} We will denote by
\begin{align*}
    \partial_x &= \mathrm{d}\Sigma ( \partial_x ) \\
    \partial_y &= \mathrm{d}\Sigma ( \partial_y ) 
\end{align*}
the push by \( \Sigma \) of the standard vector field on \( \mathbb{R}^2 \).

\begin{defn}
    \label{3DSurfaceNormal}
\hypertarget{3DSurfaceNormal}
The \textbf{surface normal vector} to the surface \( \Sigma \) at \( p \) is
\[ 
    n_p = \frac{ \partial_x \times \partial_y }{ \left \lvert\lvert \partial_x \times \partial_y  \right\rvert\rvert  }.
\]
\end{defn}

Now we can start talking about curvature!
\begin{defn}
\label{3DSurfaceCurvature}
\hypertarget{3DSurfaceCurvature}
%TODO: define T_p \Sigma and add a picture
For \( v \in T_p\Sigma \) of length \( 1 \), the \textbf{curvature} 
\[ 
    \kappa_p(v) 
\]
is equal to the plane curvature \( \kappa \) at \( p \) of the curve, which is the intersection of \( \Sigma \) and the plane
\[ 
    \Lin \left\{ n_p, v \right\}.
\]
(This is a set -- the parametrization can be recovered with aid of the inverse function theorem).
\end{defn}
Since we have defined the above notion of curvature for \( \left \lvert\lvert v \right\rvert\rvert = 1 \), we have a function
\[ 
    \kappa: S^1 \to \mathbb{R} 
\]

\subsection{Example of surface curvature}

Let us isometrically change the coordinate system so that \( p = 0 \) and take
\[ 
    T_p\Sigma = \Lin \left \{ e_1, e_2 \right \}.
\]

\paragraph{Problem.} This can be done so that in a neighbourhood of \( 0 \) the surface looks like the graph of
\[ 
    f = \frac{1}{2} \alpha x^2 + \frac{1}{2} \beta y^2 + O( x^3 + y^3 ).
\]
Let \( v = (\cos \theta, \sin \theta, 0) \). We have that
\[ 
    n_p = (0,0,1) 
\]
and so
\[ 
    \gamma_v(t) = f(tv) = \frac{1}{2}\alpha \cos^2\theta + \frac{1}{2}\beta \sin^2\theta + O(t^3) = ct^2 + O(t^3).
\]
The curvature of this after going back to the plane is the same as the curve
\[ 
    g(t) = ct^2. 
\]
What is the curvature of \( g(t) \) at \( t = 0 \)? After applying a homothety of scale \( c^{-1} \) we get a curve \( h(t) = t^2 \), which has curvature \( 2 \) at \( t = 0 \). By considering osculating circles, the curvature of \( g(t) \) is
\[ 
    2c = \alpha\cos^2\theta  + \beta\sin^2\theta.
\]
The extreme values of this are \( \alpha, \beta \) obtained for \( v = (1,0,0) \) and \( v = (0,1,0) \). Note that the vectors are orthogonal!

\subsection{Why will we multiply these?}

Picture one -- immersing a square into \( \mathbb{R}^3 \) as a cylinder.

\section{Euclidean connection}

Take two vector fields \( X, Y  \) and a surface \( \Sigma \). We have
\[ 
    D_X Y = (X(Y_1), X(Y_2), X(Y_3)). 
\]
A part of that will be tangent to \( \Sigma \), and part will be normal. Writing this decomposition down defines
\[ 
    D_X Y = \nabla_X Y + \mathrm{I\!I}(X,Y) n_p.
\]
