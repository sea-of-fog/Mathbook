\chapter{Differential Geometry, Lecture 2}

\section{Curves in \( \mathbb{R}^3 \)}

Let \( \gamma: (a,b) \to \mathbb{R}^3 \) be parametrised by arclength and \( \gamma'' \neq 0 \). Then
\[ 
    \gamma' \perp \gamma'' 
\]
and we can define
\[ 
    N := \frac{ \gamma'' }{ \left \lvert\lvert \gamma'' \right\rvert\rvert  }.
\]
We need a third vector, ideally so that the three vectors are positively oriented. No worries, we can put down
\[ 
    B := \tau \times N. 
\]

\begin{defn}
    We define the curvature in \( \mathbb{R}^3 \) to be
    \[ 
       \kappa_\gamma := \left \lvert\lvert \gamma'' \right\rvert\rvert  
   \].
\end{defn}

\begin{defn}
    We define \emph{trójnóg Freneta} to be the ordered orthonormal basis
    \[ 
        (\tau, N, B).
   \]
\end{defn}

%TODO: name of the B vector
We have an analogue of the Frenet equations
\begin{thm}[3D Frenet Equations]
\label{FrenetEquations3D}
\hypertarget{FrenetEquations3D}
    For the \emph{Frenet trójnóg} we have
    \[ 
       \frac{\mathrm{d}^{}}{\mathrm{d}^{}t} (T, N, B) = (T,N,B) M
       %TODO
   \]
\end{thm}

\begin{proof}
The first row follows from the definition. Let
\[ 
    A = (T,N,B). 
\]
This is an orthonormal matrix for all \( t \in (a,b) \), so it traces a curve in \( \SO(3) \) and we want its derivative. We will show
\[ 
    A^{-1}  A'
\]
is skew-symmetric -- this will give us the theorem. We know that
\[ 
   AA^T = \Id,
\]
so let us differentiate both sides and obtain
\[ 
    (A')^TA + A^TA' = 0
\]
and
\[ 
    (A^TA')^T = (A')^TA = -A^TA'. 
\]
Where we have used the matrix Leibniz rule.
%TODO: matrix Leibniz rule
\end{proof}

\begin{defn}
    The function \( \tau \) mentioned in the previous theorem is called \textbf{torsion}.
\end{defn}

We have another analogue to curves in \( \mathbb{R}^2 \).

\begin{thm}[Fundamental Theorem of Curves is \( \mathbb{R}^3 \)]
\label{FundamentalCurveTheorem3D}
\hypertarget{FundamentalCurveTheorem3D}
For a positive function
\[ 
    \kappa: (a,b) \to \mathbb{R}_+,
\]
an arbitrary function
\[ 
    \tau: (a,b) \to \mathbb{R} 
\]
and any positively oriented orthonormal basis of \( \mathbb{R}^3 \) (the initial Frenet frame), there exists exactly one arclength parametrised curve
\[ 
    \gamma: (a,b) \to \mathbb{R}^3
\]
which has Frenet frame at \( t = 0 \) equal to \( B \).
\end{thm}

\begin{proof}[Proof (sketch).]
As in the 2-dimensional case -- put down a differential equation.
\end{proof}

\paragraph{What is torsion?} The torsion describes how much a curve escapes a plane in which it starts (the \emph{osculating plane}).

\section{Curvature of surfaces}

% TODO: inclusion arrow
Consider a surface 
\[ 
    S \subseteq \mathbb{R}^3.
\]
We will want to define its curvature in terms of plane curvature. We will work with surfaces given as immersions
\[ 
    \Sigma: U \to \mathbb{R}^3,
\]
where \( U \subseteq \mathbb{R}^2 \).

\begin{defn}
    Let \( p \in U \subseteq \mathbb{R}^2 \) and \( \Sigma \) be as above. The \textbf{Riemann metric} on \( TU \) is the smooth, positive definite bilinear form given at \( p \) by
    \[ 
        \left\langle v_p, w_p \right\rangle_\Sigma = \left\langle \mathrm{d}\Sigma(v_p), \mathrm{d}\Sigma(w_p) \right\rangle. 
   \]
\end{defn}

\paragraph{Notation.} We will denote by
\begin{align*}
    \partial_x &= \mathrm{d}\Sigma ( \partial_x ) \\
    \partial_y &= \mathrm{d}\Sigma ( \partial_y ) 
\end{align*}
the push by \( \Sigma \) of the standard vector field on \( \mathbb{R}^2 \).

\begin{defn}
    \label{3DSurfaceNormal}
\hypertarget{3DSurfaceNormal}
The \textbf{surface normal vector} to the surface \( \Sigma \) at \( p \) is
\[ 
    n_p = \frac{ \partial_x \times \partial_y }{ \left \lvert\lvert \partial_x \times \partial_y  \right\rvert\rvert  }.
\]
\end{defn}

Now we can start talking about curvature!
\begin{defn}
\label{3DSurfaceCurvature}
\hypertarget{3DSurfaceCurvature}
%TODO: define T_p \Sigma and add a picture
For \( v \in T_p\Sigma \) of length \( 1 \), the \textbf{curvature} 
\[ 
    \kappa_p(v) 
\]
is equal to the plane curvature \( \kappa \) at \( p \) of the curve, which is the intersection of \( \Sigma \) and the plane
\[ 
    \Lin \left\{ n_p, v \right\}.
\]
(This is a set -- the parametrization can be recovered with aid of the inverse function theorem).
\end{defn}
Since we have defined the above notion of curvature for \( \left \lvert\lvert v \right\rvert\rvert = 1 \), we have a function
\[ 
    \kappa: S^1 \to \mathbb{R} 
\]

\subsection{Example of surface curvature}

Let us isometrically change the coordinate system so that \( p = 0 \) and take
\[ 
    T_p\Sigma = \Lin \left \{ e_1, e_2 \right \}.
\]

\paragraph{Problem.} This can be done so that in a neighbourhood of \( 0 \) the surface looks like the graph of
\[ 
    f = \frac{1}{2} \alpha x^2 + \frac{1}{2} \beta y^2 + O( x^3 + y^3 ).
\]
Let \( v = (\cos \theta, \sin \theta, 0) \). We have that
\[ 
    n_p = (0,0,1) 
\]
and so
\[ 
    \gamma_v(t) = f(tv) = \frac{1}{2}\alpha \cos^2\theta + \frac{1}{2}\beta \sin^2\theta + O(t^3) = ct^2 + O(t^3).
\]
The curvature of this after going back to the plane is the same as the curve
\[ 
    g(t) = ct^2. 
\]
What is the curvature of \( g(t) \) at \( t = 0 \)? After applying a homothety of scale \( c^{-1} \) we get a curve \( h(t) = t^2 \), which has curvature \( 2 \) at \( t = 0 \). By considering osculating circles, the curvature of \( g(t) \) is
\[ 
    2c = \alpha\cos^2\theta  + \beta\sin^2\theta.
\]
The extreme values of this are \( \alpha, \beta \) obtained for \( v = (1,0,0) \) and \( v = (0,1,0) \). Note that the vectors are orthogonal!

\subsection{Why will we multiply these?}

Picture one -- immersing a square into \( \mathbb{R}^3 \) as a cylinder.

\section{Euclidean connection}

Take two vector fields \( X, Y  \) and a surface \( \Sigma \). We have
\[ 
    D_X Y = (X(Y_1), X(Y_2), X(Y_3)). 
\]
A part of that will be tangent to \( \Sigma \), and part will be normal. Writing this decomposition down defines
\[ 
    D_X Y = \nabla_X Y + \mathrm{I\!I}(X,Y) n_p.
\]

\begin{lemma}[Propeties of \( \nabla \)]
\label{EuclideanConnectionPropeties}
\hypertarget{EuclideanConnectionPropeties}
% TODO: add any two
For any surface \( \Sigma \), vector fields \( X, Y, Z \) and function \( f \) we have
\begin{enumerate}
    \item \( \nabla_{fX}Y = f\nabla_X Y \)
\item \(\nabla \emph{agrees with} \left\langle -,\, - \right\rangle \), i.e.
    \[ 
       X \left\langle Y,Z \right\rangle = \left\langle \nabla_X Y, Z \right\rangle + \left\langle Y, \nabla_X Z \right\rangle. 
   \]
\item the connection \( \nabla \) is torsion-free, i.e.
    \[ 
        \nabla_X Y - \nabla_Y X = [X, Y],
   \]
\item the second fundamental form is symmetric and bilinear, i.e.
    \[ 
        \mathrm{I\!I} (fX, Y) = f\mathrm{I\!I}(X,Y)
   \]
   \[ 
       \mathrm{I\!I}(X,Y) = \mathrm{I\!I}(Y,X). 
  \]
\end{enumerate}
\end{lemma}

\begin{proof}
    For the first one,
    \[ 
        D_{fX} Y = fD_XY = f\nabla_X Y + f\mathrm{II}(X,Y)n.
   \]
   For the second one
   \[ 
       D_X(fY) = X(f)Y + fD_X Y = X(f)Y + f\nabla_X Y + f \mathrm{II}(X,Y) n. 
  \]
  For (2)
  \[ 
     X \left\langle Y,Z \right\rangle = X \left( \sum_i Y_i Z_i \right) = \sum_i X(Y_i)Z_i + Y_iX(Z_i) = \left\langle D_X(Y), Z \right\rangle + \left\langle Y, D_X(Z) \right\rangle = \left\langle \nabla_X Y, Z \right\rangle + \left\langle Y, \nabla_X Z \right\rangle.
 \]
 For (3), we know that
\[ 
    [X, Y]f = X(Yf) - Y(Xf)
\]
Let us extend \( X, Y \) to vector fields \( \widetilde{X}, \widetilde{Y} \). Then
\[ 
    D_X Y - D_Y X = D_{\widetilde{X}} \widetilde{Y} - D_{\widetilde{Y}} \widetilde{X} = [\widetilde{X}, \widetilde{Y}] = [X, Y].  
\]
\end{proof}
This gives some corollaries.
\begin{enumerate}
    \item The value of \( (D_XY)_p \) depends only on \( X_p, Y_p \) and the first derivaties of \( Y \). Therefore the same is true for \( \nabla_X Y \) and \( \mathrm{I\!I} (X, Y) \).
    \item Since \( \mathrm{I\!I} \) is symmetric, the value of \( \mathrm{I\!I}(X, Y)_p \) depends only on \( X_p \) and \( Y_p \). And particular, \( \mathrm{I\!I} \) is a bilinear form on \( T_p \Sigma \).
\end{enumerate}

\begin{defn}
\label{LinearConnection}
\hypertarget{LinearConnection}
A \textbf{linear connection} is an \( \mathbb{R} \)-bilinear map
%TODO: substitute M for vector fields on M
\[ 
    \nabla: M \times M \to M 
\]
such that for any two vector fields and a smooth function \( f \in C^\infty(M) \) it holds that
\begin{enumerate}
    \item \( \nabla_{fX} Y = f\nabla_X Y \) and
    \item \( \nabla_X fY = X(f)Y + f\nabla_X Y \).
\end{enumerate}
\end{defn}

\begin{defn}
\label{LinearConnectionConsistentWithMetric}
\hypertarget{LinearConnectionConsistentWithMetric}
A linear connection is \textbf{consistent with metric} if
\[ 
    X \left\langle Y, Z \right\rangle = \left\langle \nabla_X Y, Z \right\rangle + \left\langle Y, \nabla_X Z \right\rangle. 
\]
\end{defn}

\begin{defn}
\label{LinearConnectionTorsionfree}
\hypertarget{LinearConnectionTorsionfree}
A linear connection \( \nabla \) is \textbf{torsion-free} if
\[ 
    \nabla_X Y - \nabla_Y X = [X,Y]. 
\]
\end{defn}

\begin{thm}[Fundamental Theorem of Differential Geometry]
\label{FundamentalTheoremDifferentialGeometry}
\hypertarget{FundamentalTheoremDifferentialGeometry}
Let \( M \) be a Riemannian manifold. Then there exists exactly one linear connection that is torsionfree and consistent with metric. It is called the \textbf{Levi-Civita connection}.
\end{thm}

\begin{defn}
\label{RiemannianManifold}
\hypertarget{RiemannianManifold}
Let \( M \) be smooth manifold and \( TM \) be its tangent bundle. A \textbf{Riemann metric} on \( M \) is a smooth, bilinear, symmetric, positive definite form
\[ 
    g_p = \left\langle -,\, - \right\rangle_p : T_pM \times T_pM \to \mathbb{R}. 
\]
Smoothness means that for a chart \( \varphi \), we have a basis of \( TM \) around \( p \) called \( \partial_i \) and for that basis all the functions
\[ 
    g_{ij}(p) = g_p(\partial_i(p), \partial_j(p)).
\]
\end{defn}

\paragraph{Example.} (Pullback of Riemann metric). Let \( \Sigma: U \to \mathbb{R}^3 \). Then for \( p \in U \) we can take
\[ 
    g_p : T_pU \times T_pU \to \mathbb{R} 
\]
by putting down
\[ 
    g_p(v, w) = \left\langle D\Sigma(v), D\Sigma(w) \right\rangle  
\]
where \( \left\langle -,\, - \right\rangle  \) is the \( \mathbb{R}^3 \) dot product.
 
\paragraph{Example (hyperbolic plane).} Take \( \mathbb{H}^2 = \mathbb{R} \times \mathbb{R}_+ \). We can give a Riemann Metric by
\[ 
    \left\langle (v_1, v_2), (w_1, w_2) \right\rangle  = \frac{v_1w_1 + v_2w_2}{y^2}.
\]

Why is this called a \emph{Riemann Metric}? Well, since we can measure tangent vectors, we can measure the \emph{length} of curves!
\[ 
    d( \gamma ) = \int\limits_{0}^{1} \left\langle \dot{\gamma}(t), \dot{\gamma}(t) \right\rangle_{\gamma(t)} \,\mathrm{d}t.
\]
Now, we can measure the distance by
\[ 
    d_g(x,y) = \inf d(x). 
\]

\begin{proof}[Proof of the Fundamental Theorem of Differential Geometry.]
    Existence work by partitions of unity. For uniqueness, we will use
    \begin{align*}
        X \left\langle Y, Z \right\rangle &= \left\langle \nabla_X Y, Z \right\rangle + \left\langle Y, \nabla_X Z \right\rangle \\
        Y \left\langle Z, X \right\rangle &= \left\langle \nabla_Y Z, X \right\rangle + \left\langle Z, \nabla_Y X \right\rangle \\
        -Z \left\langle X, Y \right\rangle &= - \left\langle \nabla_Z X, Y \right\rangle - \left\langle X, \nabla_Z Y \right\rangle.
    \end{align*}
% Adding these three equations up
% \[ 
%     X \left\langle Y, Z \right\rangle + Y \left\langle Z, X \right\rangle - Z \left\langle X, Y \right\rangle =

% \]
\end{proof}
