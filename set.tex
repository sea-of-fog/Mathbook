\chapter{Naive set theory}

% TODO: add a tree picture
\begin{lemma}
    Let
    \[
        f: X \to X
    \]
    be a function from a \emph{finite} nonempty set into itself. Then, you can uniquely decompose the set \( X \) into disjoint parts
    \[
        X = X_1 \cup X_2
    \]
    such that \( f \) restricted to \( X_1 \) is a permutation and every element from \( X_2 \) eventually gets to \( X_1 \), that is
    \[
        \forall x_2 \in X_2 \exists k \, f^k \in X_1.
    \]
\end{lemma}

\begin{proof}

    Define a decreasing (TODO: why) sequence of sets by
    \[
        X^0 = 0, X^{k+1} = f[X^k].
    \]
    Every set in the sequence is finite and nonempty. Since the sequence is decreasing, it must stabilize eventually. Let \( k_0 \) be the first index at which it stabilizes, i.e.
    \[
        X^k = X^{k+1} = f[X^k].
    \]
    Then
    % TODO: add restriction
    \[
        f: X^k \to X^k
    \]
    % TODO: add lemma reference
    is surjective, so by a previous lemma it is bijective. We take
    \[
        X_1 = X^k,\,X_2 = X \setminus X^k
    \]
    to form the desired decomposition.

\end{proof}

\paragraph{Visualization} If we want to visualize how such a function looks, let us first recall a visualization of permutations -- they form cycles! The elemnts outside of the \emph{permutation set} \(X_1\) form trees, whose roots are vertices of \( X_1 \).

% TODO: examples
\begin{example}
    Consider the function \ldots
\end{example}

This may seem like a silly little fact, but the ideas come up often: in the Lefschetz Fixpoint theorem, Jordan decomposition from linear algebra \ldots

\paragraph{How do we extend this to infinite sets?} Let's try using a fixpoint theorem! In this case, we'll be looking for the largest fixpoint of a decreasing function.
Note that on infinite sets the trees mentioned earlier may be infinite (consider the example of \( x \mapsto x+1 \) on \(\mathbb{N}\)). But these are actually permutations! So maybe the structure looks the same? But how do we define the \emph{permutation set}?

Attempt 1: take the maximal set from
\[
    \{ A \mid f \text{ is a permutation on } A \}
\]

\section{Useful identities on sets}

\begin{lemma}
\label{SumOfDifferences}
\hypertarget{SumOfDifferences}
For any two families \( A_\alpha \), \( B_\alpha \) indexed by \( \alpha \in \mathcal{A} \), we have
\[ 
    \bigcup_{\alpha \in \mathcal{A}} A_\alpha \setminus \bigcup_{\alpha \in \mathcal{A}} B_\alpha  \subseteq \bigcup_{\alpha \in \mathcal{A}} (A_\alpha \setminus B_\alpha)
\]

\end{lemma}

\section{Literature used for this chpater}
