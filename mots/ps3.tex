\chapter{Measures on Topological Spaces, Problemset 3}

\section*{Problem 1}

%TODO: separate into a lemma
By transfinite induction, each \( B \in \Bor Y \) is of the form \( \widetilde{B} \cap Y \) for some \( \widetilde{B} \in \Bor X \). 

The axiom \( \nu( \varnothing ) = 0 \) is immediate from \( \varnothing = \varnothing \cap Y \).

For countable additivity, take a sequence of pairwise disjoint sets \( B_n \in \Bor Y \). By an earlier observation, we may represent them as \( Y \cap \widetilde{B}_n \) for a sequence \( \widetilde{B}_n \in \Bor X \). By Problem 2, we may in fact assume
\[ 
    \mu(\widetilde{B}_n) = \mu^*(Y \cap B_n)
\]
if we take \( \widetilde{B}_n \) to be measurable hulls. If \( B_n, B_m \) give disjoint sets in \( Y \), then we have
\[ 
    \widetilde{B}_n \cap \widetilde{B}_m \subseteq (\widetilde{B}_n \setminus Y) \cup (\widetilde{B}_m \setminus Y)
\]
so when we pass to outer measure we see that \( \mu(\widetilde{B}_n \cap \widetilde{B}_m) = 0. \) Note that \( B_n \cap Y = \widetilde{B}_n \cap Y \). We have
\[ 
    0 \leqslant \mu^* \left( \bigcup_{n=1}^\infty \widetilde{B}_{n} \setminus Y\right) \leqslant \sum_{n=1}^\infty \mu^*(\widetilde{B}_n \setminus Y) = 0,
\]
so
\begin{align*}
    \mu^* \left( \bigcup_{n=1}^\infty \widetilde{B}_n \cap Y \right) &= \mu^* \left( \bigcup_{n=1}^\infty \widetilde{B}_n \right) \\
&= \mu \left( \bigcup_{n=1}^\infty \widetilde{B}_n \right) \\
&= \sum_{n=1}^\infty \mu \left( \widetilde{B}_n \right) \\
&= \sum_{n=1}^\infty \mu^* \left( \widetilde{B}_n \cap Y \right).
\end{align*}
\section*{Problem 2}

By definition of outer measure (as an infimum), we can choose measurable sets \( H_n \supseteq Z \) such that
\[ 
    \mu^*(Z) \leqslant \mu(H_n) < \mu^*(Z) + \frac{1}{n}. 
\]
Take
\[ 
    H = \bigcap_{n=1}^\infty H_n. 
\]
This \( H \) contains \( Z \), so we have
\[ 
    \mu(H) = \mu^*(Z) 
\]
by squeezing. By regularity of borel measures (see \ref{FirstRegularityTheorem}), we can take a \( G_\delta \) upper approximation of \( H \) with the same measure.

For the second part, take two measurable hulls \( H_1 \) and \( H_2 \). Since \( Z \subseteq H_1 \cap H_2 \), we have
\[ 
    H_1 \Delta H_2 \subseteq (H_1 \setminus Z) \cup (H_2 \setminus Z),
\]
and the RHS has (outer) measure \( 0 \), so the LHS does as well.

\paragraph{Observation.} Hulls work well with set unions, i.e. for a countable union of \( Z_i \) with hulls \( H_i \), the union of \( H_i \) is a hull for that union. Intersections and complements are more problematic.

\section*{Problem 3}

By a \emph{base} I understand a family \( \mathcal{U} \) of subsets of \( X \) such that \( \sigma( \mathcal{U} ) = \Bor X \). Consider the family \( \mathcal{A} \) of borel subsets of \( X \) on much \( \mu \) and \( \nu \) agree. We will show that it is a \( \sigma \)-algebra.

Because these are probability measures, the family \( \mathcal{A} \) is closed under complements. By upward continuity of measures, \( \mathcal{A} \) is closed under finite unions. Therefore it is a \( \sigma \)-algebra, so \( \mathcal{A} = \Bor X \) and \( \mu = \nu \).

\subsection*{Cardinality}

Since \( X \) has at least two distinct points \( x_1, x_2 \), we have at least \( \mathfrak{c} \) measures, as witnessed by
\[ 
    p\delta_{x_1} + (1-p)\delta_{x_2}. 
\]
On the other hand, we have
\begin{lemma}
\label{SMSpaceSecondCountable}
\hypertarget{SMSpaceSecondCountable}
If \( X \) is \( \mathcal{SM} \), then \( X \) is second countable.
\end{lemma}

\begin{proof}
    By \ref{SMSpaceEmbedsInHilbertCube}, \( X \hookrightarrow [0,\,1]^{ \mathbb{N} } \) which is second countable, so \( X \) is second countable as well. 
\end{proof}

Since the values of a probability measure \( \mu \) are determined by its values on a countable basis, we know that there are at most as many measures as functions in \( [0,\,1]^{ \mathbb{N} } \). The cardinality of the Hilbert Cube is
\[ 
    \mathfrak{c}^{\aleph_0} = \left( 2^{\aleph_0} \right)^{\aleph_0} = 2^{\aleph_0 \cdot \aleph_0} = 2^{\aleph_0} = \mathfrak{c}, 
\]
which gives the upper bound.

\paragraph{Question.} What is the finite union assumption needed for?
