\chapter{Measures on Topological Spaces, Problemset 3}

\section*{Problem 1}

%TODO: separate into a lemma
By transfinite induction, each \( B \in \Bor Y \) is of the form \( \widetilde{B} \cap Y \) for some \( \widetilde{B} \in \Bor X \). 

The axiom \( \nu( \varnothing ) = 0 \) is immediate from \( \varnothing = \varnothing \cap Y \).

For countable additivity, take a sequence of pairwise disjoint sets \( B_n \in \Bor Y \). By an earlier observation, we may represent them as \( Y \cap \widetilde{B}_n \) for a sequence \( \widetilde{B}_n \in \Bor X \). By Problem 2, we may in fact assume
\[ 
    \mu(\widetilde{B}_n) = \mu^*(Y \cap B_n)
\]
if we take \( \widetilde{B}_n \) to be measurable hulls. If \( B_n, B_m \) give disjoint sets in \( Y \), then we have
\[ 
    \widetilde{B}_n \cap \widetilde{B}_m \subseteq (\widetilde{B}_n \setminus Y) \cup (\widetilde{B}_m \setminus Y)
\]
so when we pass to outer measure we see that \( \mu(\widetilde{B}_n \cap \widetilde{B}_m) = 0. \) Note that \( B_n \cap Y = \widetilde{B}_n \cap Y \). We have
\[ 
    0 \leqslant \mu^* \left( \bigcup_{n=1}^\infty \widetilde{B}_{n} \setminus Y\right) \leqslant \sum_{n=1}^\infty \mu^*(\widetilde{B}_n \setminus Y) = 0,
\]
so
\begin{align*}
    \mu^* \left( \bigcup_{n=1}^\infty \widetilde{B}_n \cap Y \right) &= \mu^* \left( \bigcup_{n=1}^\infty \widetilde{B}_n \right) \\
&= \mu \left( \bigcup_{n=1}^\infty \widetilde{B}_n \right) \\
&= \sum_{n=1}^\infty \mu \left( \widetilde{B}_n \right) \\
&= \sum_{n=1}^\infty \mu^* \left( \widetilde{B}_n \cap Y \right).
\end{align*}
\section*{Problem 2}

By definition of outer measure (as an infimum), we can choose measurable sets \( H_n \supseteq Z \) such that
\[ 
    \mu^*(Z) \leqslant \mu(H_n) < \mu^*(Z) + \frac{1}{n}. 
\]
Take
\[ 
    H = \bigcap_{n=1}^\infty H_n. 
\]
This \( H \) contains \( Z \), so we have
\[ 
    \mu(H) = \mu^*(Z) 
\]
by squeezing. By regularity of borel measures (see \ref{FirstRegularityTheorem}), we can take a \( G_\delta \) upper approximation of \( H \) with the same measure.

For the second part, take two measurable hulls \( H_1 \) and \( H_2 \). Since \( Z \subseteq H_1 \cap H_2 \), we have
\[ 
    H_1 \Delta H_2 \subseteq (H_1 \setminus Z) \cup (H_2 \setminus Z),
\]
and the RHS has (outer) measure \( 0 \), so the LHS does as well.

\paragraph{Observation.} Hulls work well with set unions, i.e. for a countable union of \( Z_i \) with hulls \( H_i \), the union of \( H_i \) is a hull for that union. Intersections and complements are more problematic.

\section*{Problem 3}

By a \emph{base} I understand a basis for the topology, in particular \( \sigma( \mathcal{U} ) = \Bor X \). First, we will show that \( \mu = \nu \) on open sets. Take an open set \( U \). It can be represented as
\[ 
    U = \bigcup_{n=1}^\infty U_n = \bigcup_{k=1}^\infty \bigcup_{n=1}^k U_n
\]
%TODO: pi-\lambda lemma
for some \( U_n \in \mathcal{U} \). Since \( \mathcal{U} \) is closed under finite sums, the inner sums are also in \( \mathcal{U} \), so \( \mu \) and \( \nu \) agree on them. But since the outer sums on RHS are increasing, \( \mu(U) = \nu(V) \). Applying the \( \pi-\lambda \) Lemma shows that \( \mu \) and \( \nu \) agree on all Borel sets.

Alternatively, since these are probability measures, they also agree on closed sets (by complements), so the regularity property \ref{FirstRegularityTheorem} does the job.

\subsection*{Cardinality}

Since \( X \) has at least two distinct points \( x_1, x_2 \), we have at least \( \mathfrak{c} \) measures, as witnessed by
\[ 
    p\delta_{x_1} + (1-p)\delta_{x_2}. 
\]
On the other hand, we have
\begin{lemma}
\label{SMSpaceSecondCountable}
\hypertarget{SMSpaceSecondCountable}
If \( X \) is \( \mathcal{SM} \), then \( X \) is second countable.
\end{lemma}

\begin{proof}
    By \ref{SMSpaceEmbedsInHilbertCube}, \( X \hookrightarrow [0,\,1]^{ \mathbb{N} } \) which is second countable, so \( X \) is second countable as well. 
\end{proof}

Since the values of a probability measure \( \mu \) are determined by its values on a countable basis, we know that there are at most as many measures as functions in \( [0,\,1]^{ \mathbb{N} } \). The cardinality of the Hilbert Cube is
\[ 
    \mathfrak{c}^{\aleph_0} = \left( 2^{\aleph_0} \right)^{\aleph_0} = 2^{\aleph_0 \cdot \aleph_0} = 2^{\aleph_0} = \mathfrak{c}, 
\]
which gives the upper bound.

\section*{Problem 4}

% TODO: Egorov's Theorem
Let \( \mathcal{F} \) be the family of functions which have this property. It is of course closed under uniform limits, but because of the regularity property of Borel measures (see \( \ref{FirstRegularityTheorem} \)) and Egorov's Theorem it is also closed under almost uniform limits, so also under pointwise limits.

\section*{Problem 5}

We will first state and prove useful propeties of the Baire space \( \omega^\omega \).

\begin{lemma}
\label{BaireSpaceSurjectsPolishSpace}
\hypertarget{BaireSpaceSurjectsPolishSpace}
Every Polish space is a continuous image of the Baire Space \( \omega^\omega \).
\end{lemma}
