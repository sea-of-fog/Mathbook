\chapter{Measures on Topological Spaces, Problemset 1}

\section*{Problem 4}


\subsection*{Extension 1}
We show that the set can be the graph of a function! Let \(Z\) be a borel set of positive measure and define
\[
    T_Z = \{ x : \lambda(Z_x) > 0 \}.
\]
Then \(T_Z\) is a measurable set by Fubini's Theorem. We can pick a compact subset \( T'_Z \). A compact set of positive measure has at least \(\mathfrak{c}\) elements, and there are as many borel sets. Then, enumerate borel sets of \( \mathbb{R}^2 \).

\section*{Problem 5}

\subsection*{Set of undefined density at \( 0 \)}
\textbf{TODO} 

\subsection*{Set of density \( t \) at \( 0 \)}

Presented in class by \textbf{Michał Baran}. 
Fix \( t \in (0,\,1) \). 

The set we will construct will be symmetric around 0. We will find a sequence \( b_n \) such that with
\[ 
    A_n = \left( \frac{1}{n} - b_n,\, \frac{1}{n} \right) 
\]
we will have for all \( n \)
\[ 
    \frac{t}{n} =\lambda \left( \bigcup_{k = n}^\infty A_n \right) = \sum_{k = n}^{\infty} b_k,
\]
so
\[ 
    b_n = \sum_{k=n}^\infty b_k  - \sum_{k=n+1}^\infty b_k = \frac{t}{n(n+1)}.
\]
Consider
\[ 
    A :=  \bigcup_{k=1}^\infty A_k \cup -A_k.
\]
We will bound the fraction
\[ 
    \frac{\lambda \left( A \cap (-\delta, \delta) \right) }{2\delta} = \frac{\lambda \left( A \cap (0, \delta)  \right) }{\delta}
\]
from above and below. For \( \delta \in (1/(n+1), 1/n] \) we have
\[ 
    \bigcup_{k=n+1} A_k \subseteq A \cap (0, \delta) \subseteq \bigcup_{k=n} A_k,
\]
passing to measure
\[ 
    \frac{t}{n+1} \leqslant\lambda \left( A \cap (0, \delta) \right) \leqslant \frac{t}{n}.
\]
When divided by \( \delta \), we get the result by the squeeze theorem.

\paragraph{Remark.} The solution would work equally well if instead of \( a_n = 1/n \) we used a sequence that converges to \( 0 \) monotonically and satisfies
\[ 
    \frac{a_n - a_{n+1}}{a_n} \to 0.
\]

\section*{Problem 9}

Presented in class by \textbf{dr Arturo Martinez Celiz}.

Wlog, everything happens within \( (0,\,1) \). Following the hint, take a countable sequence \( A_i \) such that the set \( B := \bigcup_i A_i \) has maximal measure.

By this choice, for any \( C \in \mathcal{A} \), we have
\[ 
   \lambda \Bigl( \left( C \cup B \right) \Delta B \Bigr) = 0, 
\]
so that
\[ 
    \phi(C \cup B) = \phi(B)
\]
and
\[ 
    C \subseteq \phi(C \cup B) = \phi(B). 
\]
Since \( C \) was arbitrary
\[ 
    \bigcup \mathcal{A} \subseteq \phi(B) \implies B \subseteq \bigcup \mathcal{A} \phi(B).
\]
Since \( \lambda(B) =\lambda \left( \phi(B) \right) \) we know that the sum of \( \mathcal{A} \) is measurable.

\section*{Problem 10}

Presented in class by \textbf{Szymon Smolarek}. 

We take a cover of \textbf{regular sets}, i.e. a family for which there exists a constant \( C \) such that
\[ 
    \diam^2 A \leqslant C \lambda_2(A).
\]
It can be proven that if such a family is a Vitali cover of a set \( A \subseteq \mathbb{R}^2 \), an analogue of the \hyperlink{VitaliCoveringTheorem}{VCT} holds.

The family of all triangles does not satisfy the regularity condition -- think of keeping one segment constant and bringing the third verted evert closer to the segment. To deal with this, we subdivide the family \( \mathcal{T} \) into subfamilies
\[ 
    \mathcal{T}_n := \left\{ T \in \mathcal{T} : \diam^2 T \leqslant n\lambda_2(A) \right\}.
\]
Reducing to a given subfamily, we can cover each triangle \( T \) by arbitrarily small traingles similar to \( T \) contained within \( T \). This gives us a regular Vitali cover \( \widetilde{T}_n \) of \( \bigcup \mathcal{T}_n \).

\section*{Problem 11}

Stated in class by \textbf{Szymon Smolarek}.

\begin{thm}[Steinhaus theorem for the Cantor Set]
\label{CantorSteinhaus}
\hypertarget{CantorSteinhaus}
For any measurable set \( A \), the set
\[ 
    A \oplus A 
\]
contains an open neighbourhood of \( 0 \).
\end{thm}

\begin{thm}[Vitali Covering Theorem for the Cantor Set]
\label{CantorSetVitaliCoveringTheorem}
\hypertarget{CantorSetVitaliCoveringTheorem}
    If a family of clopens \( \mathcal{J} \subseteq \Clop \mathcal{C} \) is a Vitali cover of \( A \), then there is a sequence \( J_n \in \mathcal{J} \) such that
    \[ 
      \nu^* \left( A \setminus \bigcup_n J_n \right) = 0. 
   \]
\end{thm}

\begin{thm}[Lebesgue Density Theorem for the Cantor Set]
\label{CantorSetLebesgueDensityTheorem}
\hypertarget{CantorSetLebesgueDensityTheorem}
    Let \( A \subseteq \mathcal{C} \). An element \( a \in A \) is a \textbf{density point} of \( A \) if
    \[ 
        \lim_{n \to \infty} \frac{ \nu \left( A \cap [a|_{[n]}] \right) }{2^{-n}} = 1.
   \]
   If \( A \) is measurable, then almost all points of \( A \) are density points of \( A \).
\end{thm}

The proofs are quite the same, as \( \mathcal{C} \) is a topological group and the measure \( \nu \) is its Haar measure.

\section*{Problem 12}

\paragraph{Hint.} Use Baire's theorem.
