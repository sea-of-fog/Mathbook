\chapter{Boolean algebras}

\begin{thm}
\label{MeasureAlgebraIsomorphic}
\hypertarget{MeasureAlgebraIsomorphic}
Let \( (X, \Sigma, \mu) \) be a nonatomic Borel probability space and \( \mu \) be separable (or, more scientifically, of countable type). Then there exists a Borel isomorphism
\[ 
    h: \Sigma/ \mathcal{N}(\mu) \to \Bor [0,1]/ \mathcal{N}(\lambda )
\]
that preserves the measure, i.e.
\[ 
    \lambda \circ h = \mu. 
\]
A measure is of countable type iff there is a countable family \( \mathcal{C}_1 \subseteq \Sigma \) such that for all \( A \in \Sigma \) and \(\varepsilon > 0 \) there is a \( C \in \mathcal{C}_1 \) such that
\[ 
    \mu \left( C \Delta A \right) <\varepsilon,
\]
i.e. the algebra with the Frechet-Nikodym metric is separable.
\end{thm}

\begin{proof}
    This will be a back-and-forth proof. Denote the algebras by \( \mathbb{A}, \mathbb{B} \). We shall endeavour to find the countable dense sets \( \mathbb{A}_0, \mathbb{B}_0 \) which are isomorphic via an isomorphism \( h_0 \). Fix \( \mathcal{C}_1, \mathcal{C}_2 \) -- countable dense subsets of \( \Sigma_1, \Sigma_2 \). We will express them as 
    \[ 
        \mathbb{A}_0 = \bigcup_{n=1}^\infty \mathbb{A}_n
   \]
   and we will have isomorphisms \( h_n: \mathbb{A}_n \to \mathbb{B}_n \) that extent one another. When picking a new set, some atoms will split in two and we will extend the isomorphism using the Darboux property of nonatomic measures.
\end{proof}

\paragraph{Corollary.} Informally said, there is only one nonatomic probability measure on \( \mathcal{SM} \) spaces. 

\begin{thm}
\label{LpAlgebrasIsomorphic}
\hypertarget{LpAlgebrasIsomorphic}
Let \( X \) be \( \mathcal{SM} \), \( \mu \in \mathbb{P}(X) \). Then
\[ 
    L^p(\mu) \cong L^p [0,1],
\]
i.e. they are \textbf{linearly isometric}. In particular, the first space is Banach!
\end{thm}
\begin{proof}
    % TODO: add abstraction classes in L^p
We do the standard thing. Put down
\[ 
    T( \chi ) = T (\chi_B), 
\]
where \( h(\dot{A}) = \dot{B} \). By linearity, we extend \( T \) to simple functions, i.e. it is a linear isometry between the normed spaces
\[ 
    Simp( \Sigma )  \to Simp( \Bor [0,1] ).
\]
We have a linear isometry between dense subspaces, so we have a dense isometry between whole spaces. Explicitly
\[ 
    \left \| \sum c_i \chi_{A_i} \right\|^p_p = \sum \left| c_i \right|^p \mu(A_i).
\]
\end{proof}

\paragraph{Remark.} For Polish spaces, we even have a whole isomorphisma, not only between algebras!

\paragraph{Questions we yet have to answer.} 
\begin{enumerate}
    \item There is an \( \mathcal{SM} \) space \( X \) and \( \mu \in \mathbb{P}(X) \) such that \( \mu(K) = 0 \) for all compact \( K \subseteq X \). This is the Bernstein set.
    \item The existance of an \( \mathcal{SM} \) space such that all probability measures are purely atomic.
    \item Interesting measures on nonseparable metric spaces.
    \item Does \(\lambda \) extend to all subsets of \( [0,1] \).
\end{enumerate}

\section{Recap of set theory.} An ordinal number is a set well-ordered by the 

\begin{lemma}[Ordinal number ordering]
\label{OrdinalNumberEmbedding}
\hypertarget{OrdinalNumberEmbedding}
Any two well-ordered sets are either isomorphic or one embeds as an initial segment of the other.
\end{lemma}

The set \( \omega \) is the only inifite well-ordered set without a maximal element whose all initial segments are finite.

\begin{thm}[Bernstein Set Theorem]
\label{BernsteinSetTheorem}
\hypertarget{BernsteinSetTheorem}
There exists a set \( Z \subseteq [0,\,1] \) such that for all uncountable compact sets \( K \)
\[ 
    Z \cap K \neq \varnothing \neq Z^c \cap K. 
\]
\end{thm}
\begin{proof}
There are \( \mathfrak{c} \)-many such sets \( K \) and they all have an embedded Cantor set (every uncountable Polish space does), so they are of size \( \mathfrak{c} \). Now for \( \alpha < \mathfrak{c} \) we can take
\[ 
    x_\alpha, y_\alpha \in K_\alpha \setminus \left\{ x_\beta, y_\beta : \beta < \alpha  \right\} 
\]
and
\[ 
     Z := \left\{ x_\alpha \right\}.
\]
\end{proof}

\begin{lemma}[Measure of the Bernstein Set]
\label{BernsteinSetMeasure}
\hypertarget{BernsteinSetMeasure}
We have
\[ 
   \lambda^*(Z) =\lambda^*(Z^c) = 1. 
\]
\end{lemma}
\begin{proof}
    If \(\lambda^*(Z) < 1 \), then there is a \( B \in \Bor [0,\,1] \) such that \(\lambda(B) < 1 \), so \( B^c \subseteq Z^c \) contains a compact set of positive measure, contradiction.
\end{proof}

\begin{thm}[Non-tight measure]
\label{NonTightMeasureBernsteinSet}
\hypertarget{NonTightMeasureBernsteinSet}
The Lebesgue measure restricted to the Bernstein set is not tight.
\end{thm}

\begin{proof}
The Borel substs of \( Z \) are restriction of Borel subsets. The
\[ 
    \mu(B \cap Z) =\lambda(B) 
\]
defines a measure. The Bernstein set has only countable compact subsets, and on them 
\[ 
    \mu(K) = 0. 
\]
\end{proof}

\paragraph{Remark.} The Bernstein set is even more nonmeasurable. In fact, for all nonatomic measures
\[ 
    \nu^* (Z) = \nu^* (Z^c) = 1. 
\]

We not turn to the Continuum Hypothesis.
\begin{thm}[Lusin set]
\label{LusinSet}
\hypertarget{LusinSet}
Suppose the Continuum Hypothesis. Then there exists a \textbf{Lusin set}, i.e. an uncountable \( L \subseteq [0,\,1] \) such that \( L \cap P \) for all closed nowhere dense sets \( P \).
\end{thm}

\begin{proof}
Let \( \left\{ F_\alpha : \alpha < \omega_1 \right\} \) be a list of all closed nowhere dense sets. Then we define
\[ 
    L = \left\{ x_\alpha : \alpha < \omega_1 \right\}, 
\]
where
\[ 
    x_\alpha \in [0,\,1] \setminus \left( \bigcup_{\beta < \alpha} F_\beta \cup \left\{ x_\beta : \beta < \alpha \right\} \right) \neq \varnothing.
\]
\end{proof}

The consequences of that for measure theory.

\begin{thm}
\label{LusinMeasureZero}
\hypertarget{LusinMeasureZero}
Let \( L \) be a Lusin set. Then
\begin{enumerate}
    \item Every nonatomic \( \mu \in \mathbb{P}[0,1] \) has \( \mu^*(L) = 0. \)
    \item Every \( \nu \in \mathbb{P}(L) \) is nonatomic.
\end{enumerate}
\end{thm}

\begin{proof}[Proof of (1).] Let \( \mu \in \mathbb{P}[0,1] \) be nonatomic. Then there is a sequence of nowhere dense \( F_n \subseteq [0,1] \) such that
    \[ 
       \mu \left( \bigcup F_n \right) = 1. 
   \]
Dually, \( \mu(G) = 0 \) for a dense \( G_\delta \) set \( G \). This follows from regularity -- take a measurable hull of a countable dense set.
\end{proof}

\begin{proof}[Proof of (2).]
    Ad absurdum, a measure \( \nu \in \mathbb{P}(L) \) extends to a measure on \( \mathbb{P}[0,\,1] \) which is still nonatomic, contraqdicting \( (1) \).
\end{proof}

\paragraph{Remark.} This gives that measure and category are quite orthogonal with what is understood as \emph{small}.

\begin{thm}
\label{LusinSetLebesgueqNonExtensible}
\hypertarget{LusinSetLebesgueqNonExtensible}
If there exists a Lusin set \( L \subseteq [0,\,1] \) then there are sets \( E_n \subseteq [0,\,1] \) such that \(\lambda \) does not admit an extension to a measure on
\[ 
    \sigma \left( \Bor [0,\,1], E_1, \ldots \right) 
\]
\end{thm}
