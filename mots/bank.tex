\chapter{Measure Theory Bank of Lemmas}

\begin{defn}[Outer measure]
\label{OuterMeasureDefinition}
\hypertarget{OuterMeasureDefinition}
A nonnegative set function \( \mu^*: \mathcal{P}(X) \to \mathbb{R} \) is called an \textbf{outer measure} when it satisfies the following three properties:
\begin{enumerate}
    \item it is null on the empty set: \( \mu^* (\varnothing) = 0 \),
    \item it is monotone: \( A \subseteq B \implies \mu^*(A) \leqslant \mu^*(B) \),
    \item is countably subadditive:
        \[ 
            \mu^* \left( \bigcup_{j=1}^\infty A_j \right) \leqslant \sum\limits_{j=1}^\infty \mu^* (A_j)
       \]
\end{enumerate}
\end{defn}

\begin{lemma}[Generating an outer measure]
\label{OuterMeasureFromMeasure}
\hypertarget{OuterMeasureFromMeasure}
Let \( \mu \) be a measure on the measurable space \( (X, \Sigma) \) and \( \mu^* \) be the function
\[ 
    \mu^* : \mathcal{P}(X) \to \mathbb{R}
\]
defined by
\[ 
    \mu^* (A) := \inf \left\{ \mu (B) : B \in \Sigma, A \subseteq B \right\}.
\]
Then \( \mu^* \) is the unique outer measure which extends \( \mu \). Moreover, the infimum may be taken over any \( \sigma \)-ring of sets (AM I SURE OF THIS?) that generates \( \Sigma \).
\end{lemma}

\begin{proof}
    TODO.
\end{proof}

\begin{lemma}[Generating a \( \sigma \)-algebra]
\label{SigmaAlgebraGeneration}
\hypertarget{SigmaAlgebraGeneration}
Fix a space \( X \). For any family of sets \( \mathcal{A} \), \( \sigma( \mathcal{A} ) \) can be generated by any of the following sets of operations: 
\begin{enumerate}
    \item the empty set, complements and countable unions.
    \item the empty set, complements, finite unions and increasing countable unions.
\end{enumerate}
\end{lemma}

\begin{lemma}
\label{BorelGeneration}
\hypertarget{BorelGeneration}
Let \( \mathcal{B} \) be a basis for the topology of \( X \). Then
\[ 
    \sigma \left( \mathcal{B} \right) = \Bor X.
\]
\end{lemma}

\begin{lemma}
\label{SetOperationsContinuous}
\hypertarget{SetOperationsContinuous}
The set operations and the measure taking operations are continuous with respect to the symmetric difference pseudometric.
\end{lemma}

\begin{lemma}
\label{BoundedUniformApproxBySimple}
\hypertarget{BoundedUniformApproxBySimple}
A bounded measurable function \( f \) on a measure space \( (X, \mu ) \) can be uniformly approximated by simple functions.
\end{lemma}
\begin{proof}
    Let
    \[ 
        \left\lvert f \right\rvert \leqslant [-M, M]. 
   \]
    We will construct the approximation by considering \emph{bins} of the values of \( f \), i.e. the sets
    \[ 
        A_k := f^{-1} \Bigl [ [k\varepsilon, (k+1) \varepsilon) \Bigr ]
   \]
   for \( k \in \mathbb{Z} \). In such a \emph{bin}, all the values are within an \(\varepsilon \) of each other. Since \( f \) is bounded, all but finitely many of the bins are empty \( X \), so the function
   \[ 
       \widetilde{f}_\varepsilon := \sum_{A_k \neq \varnothing} (k\varepsilon) \cdot \chi_{A_k}
  \]
\end{proof}
\paragraph{Remark.} This works equally well for almost everywhere bounded functions, giving almost everywhere uniform convergence.

\begin{lemma}
    \label{CuttingOutMeasure}
    \hypertarget{CuttingOutMeasure}
    Let \(A\), \(A_1\), \ldots, \(A_k\) be measurable sets such that
    \[
        \forall k.\, \mu(A_i \cap A) \geq (1 - \delta_i)\mu(A).
    \]
    Then
    \[
        \mu(A \cap A_1 \cap A_2 \cap \ldots \cap A_k) \geq \left(1 - \sum \delta_i \right) \mu(A).
    \]
\end{lemma}
\begin{proof}
    Union bound on the sets
    \[
        A \cap A_i^c.
    \]
\end{proof}

\paragraph{Remark.} This also works for an infinite sequence of sets \( A_k \); we obtain.
\[ 
    \mu \left( A \cap \bigcap_k A_k \right) \geqslant \left( 1 - \sum_k \delta_k \right)\mu(A). 
\]
