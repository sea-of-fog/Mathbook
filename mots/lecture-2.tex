\chapter{One (Cantor) set to rule them all}
\section{Ternary Cantor}
Let us begin by making a construction. Take the closed interval \( C_0 := [0,1] \) and remove the middle one third of it in such a way that the remaining two interval are closed. The result of this is
\[ 
    C_1 := \left[0, \frac{1}{3}\right] \cup \left[ \frac{2}{3}, 1 \right].
\]
Now, repeat the operation of cutting out the middle third and call the result \( C_2 \). We can repeat this \emph{ad inifinitum} and obtain a decresing sequence of sets
\[ 
    [0,1] = C_0 \supset C_1 \supset C_2 \supset \ldots.
\]
Perhaps surprisingly, there are numbers which are not removed at any step, i.e. the intersection
\[ 
    \mathcal{C}_3 := \bigcap_{k=0}^\infty C_k
\]
is nonempty! It contains \( 0 \) and \( 1 \). In fact, any number which can be written in base \( 3 \) using only 0's and 2's is an element of this intersection. These are in fact all such numbers. We introduce a tool to prove that.

\begin{lemma}
\label{PositionalSystemPrefix}
\hypertarget{PositionalSystemPrefix}
Let \( b \geqslant 2\) be a positional system base and \( x_0 \) be a number with \( k \) digits after the positional point. Then, the numbers formed by adjoining (perhaps infinitely many) digits to the base \( b \) representation of \( x_0 \) are all the numbers in the interval
\[ 
    \left[ x_0,\, x_0 + b^{-k} \right].
\]
If we allow only finite extensions, we get the \( b \)-ary rational numbers in that interval, and if we disallow the infinite extension by the digit \( (b-1) \), we get the interval
\[ 
    \left[ x_0,\, x_0 + b^{-k} \right).
\]
\end{lemma}

\begin{lemma}
\label{ZeroTwoTernaryCantor}
\hypertarget{ZeroTwoTernaryCantor}
A real number \( x \in [0,1] \) is an element of \( \mathcal{C}_3 \) iff \( x \) can be written in base using only the digits 0 and 2.
\end{lemma}

\begin{proof}[Proof (if).] 
We proceed by induction with the induction thesis: \emph{\( x \) belongs to \( \mathcal{C}_3 \) iff \( x \) can be written in base 3 so that its first k digits are 0 or 2.} It should be clear that this thesis is equivalent to the lemma statement. For each \( k \), the statement is true by \hyperlink{PositionalSystemPrefix}{the previous lemma}.
\end{proof}

\section{Abstract Cantor}

% TODO: picture of inifinite tree
The ternary Cantor set has many interesting properties. However, to study it, we will move to a more convenient representation. We can think of the Cantor set as the set of leaves of an infinite binary tree -- starting from the root, at each level we choose whether to go right or left, or whether to insert 0 or 2 as the next digit in the base-3 representation of an \( x \in \mathcal{C}_3 \).

In this way, we can represent the ternary Cantor as
\[ 
  \mathcal{C} := \left\{ 0,1  \right\}^{ \mathbb{N} }.
\]
That the map we just described is a bijection follows from
\begin{lemma}
\label{PositionalSystemCollision}
\hypertarget{PositionalSystemCollision}
Let \( d_k, \widetilde{d}_k \) be two sequences of base \( b \) digits. Then the corresponding real numbers are equal iff \( d_k \) and \( \widetilde{d} \) agree on some prefix and afterwards one of them is \( 0 \) and the other is \( (b-1) \). 
\end{lemma}
\begin{proof}
The condition implies equality of numbers by the sum of a geometric series. The other direction follows by looking at the first moment the expansions differ at then bounding the series sum.
\end{proof}

What is missing from this description is the topology. We topologize \( \mathcal{C} \) by the metric
\[ 
    d(x,y) = 
    \begin{cases}
        0 &\text{ for } x = y \\
        \frac{1}{n}&\text{ for } x \neq y,
   \end{cases}
\]
which can also be written succintly as
\[ 
    d(x,y) = \frac{1}{n_0(x,y)}
\]
with the notation
\[ 
    n_0(x,y) := \inf \left\{ n: x_n \neq y_n \right\} 
\]
for the first index at which \( x \) and \( y \) differ. The function \( d \) may not look like a metric at first sight, but in fact it has an even better property.
\begin{lemma}
\label{CantorMetricMin}
\hypertarget{CantorMetricMin}
For the metric \( d \) described above we have for all \( x, y, z \in \mathcal{C} \)
\[ 
    d(x, z) \leqslant \max \left\{ d(x, y), d(y,z) \right\}.
\]
In particular, \( d \) is an \textbf{ultrametric}.
\end{lemma}
\begin{proof}
    Recall that \( n_0(x,z) \) is the first position at which \( x \) and \( z \) differ. Then any \( y \) has to differ with at least one of \( y \) and \( z \) at \( n_0 \), but might even earlier. This gives
    \[ 
       n_0(x,z) \geqslant \min \bigl( n_0(x,y), n_0(y,z) \bigr).
   \]
   Since the function \( x \mapsto 1/x \) is decreasing, the thesis follows.
\end{proof}
We have established that \( ( \mathcal{C}, d ) \) is a metric space. It is, in fact, homeomorphic with the subspace topology of \( \mathcal{C}_3 \) inherited from \( [0,1] \).

\begin{lemma}
\label{CantorAbstractTernaryHomeo}
\hypertarget{CantorAbstractTernaryHomeo}
    The function 
    \[ 
       h_3: \mathcal{C} \to \mathcal{C}_3 
   \]
   defined by
   \[ 
       h_3(x) := \sum_{k=1}^\infty \frac{2x_k}{3^k} 
  \]
  is a homeomorphism.
\end{lemma}
\begin{proof}
    Bijectivity follows from the number-system lemma \ref{PositionalSystemCollision} and \ref{ZeroTwoTernaryCantor}. For continuity, put down \( n_0 := n_0(x,y) \) and compute
    \begin{align*}
        \left| h_3(x) - h_3(y) \right| &= \sum_{k=1}^\infty \frac{2|x_k - y_k|}{3^k} \\
                                   &\leqslant \sum_{k = n_0}^\infty \frac{2}{3^k} \\
                                   &= \frac{2}{3^{n_0}} \cdot \frac{3}{2} \\
                                   &= \frac{1}{3^{n_0 - 1}}.
    \end{align*}
    The continuity of the inverse follows from the bound
    \[ 
        \left| h_3(x) - h_3(y) \right| \geqslant \frac{2}{3^{n_0}}.
   \]
\end{proof}

The function \( h_3 \) in \ref{CantorAbstractTernaryHomeo} can be understood as a base 3 expansion operator. When we consider a base 2 expansion instead, we lose bijectivity, but we can cover the whole interval.

\begin{lemma}
\label{CantorAbstractIntervalSurjection}
\hypertarget{CantorAbstractIntervalSurjection}
The function
\[ 
    h_2: \mathcal{C} \to [0,1] 
\]
given by
\[ 
    h_2(x) := \sum_{k = 1}^\infty \frac{x_k}{2^k} 
\]
is a continuous surjection.
\end{lemma}
\begin{proof}
    Surjectivity follows from number system properties, and continuity is essentialy the same calculation as in the proof of \ref{CantorAbstractTernaryHomeo}.
\end{proof}

\begin{thm}[The Universal Property of the Cantor Set]
\label{CantorSetUniversalProperty}
\hypertarget{CantorSetUniversalProperty}
 Every metrizable compact space \( K \) is a continuout image of \( C \).
\end{thm}

\begin{proof}
Considering an element of \( C \) as a binary expansion, we have a continuous surjection
%TODO: surjection arrow
\[ 
    g: \left\{ 0,1 \right\}^{ \mathbb{N} } \to [0,1]^{ \mathbb{N} }.
\]
The space \( K \) can be embedded into the Hilbert cube. We also have a surjection
\[ 
    h: \left\{ 0,1 \right\}^{ \mathbb{N} } \to [0,1]^{ \mathbb{N} }
\]
by using the previous surjection and \emph{unweaving} the Cantor set into the product of countably many Cantor sets. The last step is restricting the embedding \( h \) to the embedded \( K \).
\end{proof}

\paragraph{A warning agains generalization.} If \( K \) is a compact set, it embeds into a \emph{Tichonow Cube}
\[ 
    K \to [0,1]^\Gamma 
\]
and we can surject the Tichonow cube with a generalized Cantor set
\[ 
    \left\{ 0,1 \right\}^{\Gamma},
\]
but the universality theorem fails!

\section{Topology of the Cantor set}

\begin{defn}[Cantor Cylinder]
\label{CantorCylinder}
\hypertarget{CantorCylinder}
Let
    \[ 
        \varphi: \mathbb{N} \partto \left\{ 0,1 \right\}
   \]
   be a partial function with finite domain.
   Then we define the \textbf{cylinder set} with base \( \phi \) as
   \[ 
       \left[ \phi \right] := \left\{ x \in \left\{ 0,1 \right\} : x\vert_I = \phi \right\}.
  \]
\end{defn}

\begin{lemma}
\label{CantorCylinderBasis}
\hypertarget{CantorCylinderBasis}
The sets \( \left[ \phi \right] \) form a base of the topology of \( \left\{ 0,1 \right\}^\mathbb{N} \).
\end{lemma}

\begin{defn}
\label{CantorSubsetDependent}
\hypertarget{CantorSubsetDependent}
A set \( A \subseteq C \) is ???? by \( I \subseteq \mathbb{N} \), which we donte by \(A \sim I\) if for all \( x \in A \), \( y \in C \) we have
\[ 
    x\vert_I = y\vert_I \implies y \in A. 
\]
Equivalently,
\[ 
    \pi^{-1}_I\pi_I [A] = A. 
\]
\end{defn}

\begin{lemma}[Clopen sets in the Cantor set]
\label{CantorClopen}
\hypertarget{CantorClopen}
A set \( A \subseteq C \) is clopen iff \( A \sim I \) for some finite \( I \subseteq \mathbb{N} \).
\end{lemma}

\begin{proof}[(Direction one)]
If \( A \) is clopen, then
\[ 
A = \bigcup_i \left[ \phi_i \right] 
\]
for some finitely many (by compactnes) \( \phi_i \) with finite domain \( I_i \). Then
\[ 
    A \sim \bigcup_i I_i. 
\]
\end{proof}

\begin{proof}[(Other direction)]
 if \( A \sim I \), blabla
\end{proof}

Immediately, a lemma follows.

\begin{lemma}[Cantor set is zerodimensional]
\label{CentorZerodimensional}
\hypertarget{CentorZerodimensional}
The Cantor set \( C \) is zerodimensional, i.e. it has a base of clopen sets.
\end{lemma}

\begin{thm}[Topological characterisation of the Cantor set]
\label{CantorSetUniversalProperty}
\hypertarget{CantorSetUniversalProperty}
If a topological space \( K \) is compact, metrizable, zerodimensional with no isolated points, then
\[ 
    K \cong C. 
\]
\end{thm}

\section{The group structure}

The Cantor set has a natural abelian group structure given by its product structure. We can phrase it even more efficiently when we think of \( C \) as \( \mathcal{P}( \mathbb{N} ) \) -- the symmetric difference (or xor for the informatically inclined).

\[ 
    A \oplus B := A \Delta B 
\]
Every element has order two!

\paragraph{Fact.} Together with the operation \( \oplus \), the Cantor set \( C \) is a compact topological group, i.e. the function
\[ 
    (x, y) \mapsto x \oplus y 
\]
is continuous (in general the second element is inversed, but here every element is its own inverse anyway). 

\section{Measure}

We can define the measure on the Cantor set as a countable product of probability measures:
\[ 
    \nu = \bigotimes_{n=1}^\infty ( \frac{1}{2} (\delta_0 + \delta_1) ).
\]
But we will do it by hand.
\begin{defn}
Let \( A \subseteq C \) be clopen. Then
\[ 
    A \sim \{1, 2, \ldots, n\} 
\]
for some \( n \). Let
\[ 
    A' := \pi_{\{1, 2, \ldots, n\} }[A]
\]We define its measure to be
\[ 
    \nu (A) := \frac{\# A'}{2^n}.
\]
\end{defn}
This makes sens with the probabilistic definition.

\begin{thm}[Well-definedness of the premeasure]
\label{CantorPremeasureWellDefined}
\hypertarget{CantorPremeasureWellDefined}
The function
\[ 
    \nu: \Clop C \to \mathbb{R} 
\]
is a well-defined, additive function on the set algebra \( \Clop C \).
\end{thm}

% TODO: add reference
\paragraph{Proof.} Since the Cantor set is compact, \( \nu \) is automatically downward continuous on the empty set. By Caratheodory's Theorem, \( \nu \) extends uniquely to a probabilistic measure on 
\[ 
    \Bor C = \sigma \left( \Clop C \right).
\]

What now??
\[ 
    \mathcal{A} := \left\{ B \in \Bor C : \forall\varepsilon > 0. \exists A \in \Clop C. \nu (A \Delta B) <\varepsilon \right\}.
\]
We prove that this is a \( \sigma \)-algebra.

\begin{thm}
\label{CantorMeasureHaar}
\hypertarget{CantorMeasureHaar}
The measure \( \nu \) is the Haar measure on \( C \), that is, the unique probability measure invariant under group actions
\[ 
    \nu \left( x \oplus B \right) = \nu (B) 
\]
for all \( x \in C \), \( B \subseteq C \).
\end{thm}
\begin{proof}
    Let us first consider \( B = \left [ \phi \right ] \), and \( I = \dom \phi \). Then
    \[ 
       \nu \left( x \oplus \left[ \phi \right] \right) = \nu \left( \left[ x \oplus \phi \right] \right) = \nu \left( \left[ \phi \right] \right).
   \]
   A clopen is a disjoint sum of \( \left[ \phi_i \right] \) for finitely many \( \phi_i \), so additivity on clopens follows. Now, take a superficially different measure
   \[ 
      \nu_x(B) := \nu \left( x \oplus B \right). 
  \]
  Since \( \nu \) and \( \nu_x \) agree on clopens, by uniqueness in Caratheodory's Theorem they agree on all sets.
\end{proof}

Note the isomorphism
\[ 
    (C, \oplus) \cong ( \mathcal{P}( \mathbb{N} ), \Delta ) 
\]
of (topological) groups.

\section{Normal number theorem}

\begin{defn}
\label{CantorTailSubset}
\hypertarget{CantorTailSubset}
Let \( A \subseteq C \). We call \( A \) a \textbf{tail} set if
\[ 
    A \sim \left\{ k : k \geqslant n \right\} 
\]
for all \( n \). Equivalently, if \( x \in A \) and \( x(n) = y(n) \) for almost all \( n \), then \( y \in A \).
\end{defn}

\paragraph{Example.} A naturally occuring example of a tail set is
\[ 
    A_\beta := \left\{ x \in C : \lim_n \frac{x(1) + \ldots x(n)}{n}= \beta \right\}.
\]

\begin{thm}[Kolmogorov zero-one law for the Cantor set]
\label{CantorZeroOneLaw}
\hypertarget{CantorZeroOneLaw}
A borel tail set \( A \subseteq C \) has measure \( 0 \) or \( 1 \).
\end{thm}

\begin{proof}
Take a basis set \( \left[ \phi \right] \). We have
\[ 
    \nu \left( \left[ \phi \right] \cap A \right) = \nu \left( \left[ \phi \right] \right) \cdot \nu(A).
\]
From this immediately follows that this work for any \( B \in \Clop C \).
%TODO: find a good notation for the Cantor set
Now approximate \( A \) by a clopen \( B \) so that
\[ 
    \nu \left( A \Delta C \right).
\]
To finish the proof, compute
\[ 
    \nu(A) \cdot \nu(B) = \nu( A \cap B ) \geqslant \nu(A) -\varepsilon\nu(A). 
\]
\end{proof}

Returning to the example we have \( \nu(A_\beta) \in \left\{ 0,1 \right\} \). We have
\[ 
    \nu(A_\beta) = \nu (A_{1-\beta}).
\]

\begin{thm}[Borel's normal number theorem]
\label{BorelNormalNumber}
\hypertarget{BorelNormalNumber}
\[ 
    \nu \left( A_{ \frac{1}{2} } \right) = 1. 
\]
\end{thm}

\paragraph{Remark.} According to Billingsley, this theorem was the founding work of modern probability theorem, which is founded on limit theorems.
