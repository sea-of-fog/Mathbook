\chapter{Cantor set}

We work with the Cantor set understood as
\[ 
    \left\{ 0,1  \right\}^{ \mathbb{N} },
\]
topologized by the metric
\[ 
    d(x,y) = 
    \begin{cases}
        0 \text{ for } x = y \\
        \frac{1}{n}\text{ for } x \neq y
   \end{cases}
\]

Note that
\begin{enumerate}
    \item it is an ultrametric
    \item it is a homeomorphism onto the base-3 Cantor set \( C \subseteq [0,1] \).
\end{enumerate}
%TODO: continuity

We begin by proving
\begin{thm}[The Universal Property of the Cantor Set]
\label{CantorSetUniversalProperty}
\hypertarget{CantorSetUniversalProperty}
 Every metrizable compact space \( K \) is a continuout image of \( C \).
\end{thm}

\begin{proof}
Considering an element of \( C \) as a binary expansion, we have a continuous surjection
%TODO: surjection arrow
\[ 
    g: \left\{ 0,1 \right\}^{ \mathbb{N} } \to [0,1]^{ \mathbb{N} }.
\]
The space \( K \) can be embedded into the Hilbert cube. We also have a surjection
\[ 
    h: \left\{ 0,1 \right\}^{ \mathbb{N} } \to [0,1]^{ \mathbb{N} }
\]
by using the previous surjection and \emph{unweaving} the Cantor set into the product of countably many Cantor sets. The last step is restricting the embedding \( h \) to the embedded \( K \).
\end{proof}

\paragraph{A warning agains generalization.} If \( K \) is a compact set, it embeds into a \emph{Tichonow Cube}
\[ 
    K \to [0,1]^\Gamma 
\]
and we can surject the Tichonow cube with a generalized Cantor set
\[ 
    \left\{ 0,1 \right\}^{\Gamma},
\]
but the universality theorem fails!

\section{Topology of the Cantor set}

% TODO: partial function arrow
\begin{defn}[Cantor Cylinder]
\label{CantorCylinder}
\hypertarget{CantorCylinder}
    Let \( I \subseteq \mathbb{N} \) be finite and
    \[ 
       \phi: I \to \left\{ 0,1 \right\}. 
   \]
   Then we define the \textbf{cylinder set} with base \( \phi \) as
   \[ 
       \left[ \phi \right] := \left\{ x \in \left\{ 0,1 \right\} : x\vert_I = \phi \right\}.
  \]
\end{defn}

\begin{lemma}
\label{CantorCylinderBasis}
\hypertarget{CantorCylinderBasis}
The sets \( \left[ \phi \right] \) form a base of the topology of \( \left\{ 0,1 \right\}^\mathbb{N} \).
\end{lemma}

\begin{defn}
\label{CantorSubsetDependent}
\hypertarget{CantorSubsetDependent}
A set \( A \subseteq C \) is ???? by \( I \subseteq \mathbb{N} \), which we donte by \(A \sim I\) if for all \( x \in A \), \( y \in C \) we have
\[ 
    x\vert_I = y\vert_I \implies y \in A. 
\]
Equivalently,
\[ 
    \pi^{-1}_I\pi_I [A] = A. 
\]
\end{defn}

\begin{lemma}[Clopen sets in the Cantor set]
\label{CantorClopen}
\hypertarget{CantorClopen}
A set \( A \subseteq C \) is clopen iff \( A \sim I \) for some finite \( I \subseteq \mathbb{N} \).
\end{lemma}

\begin{proof}[(Direction one)]
If \( A \) is clopen, then
\[ 
A = \bigcup_i \left[ \phi_i \right] 
\]
for some finitely many (by compactnes) \( \phi_i \) with finite domain \( I_i \). Then
\[ 
    A \sim \bigcup_i I_i. 
\]
\end{proof}

\begin{proof}[(Other direction)]
 if \( A \sim I \), blabla
\end{proof}

Immediately, a lemma follows.

\begin{lemma}[Cantor set is zerodimensional]
\label{CentorZerodimensional}
\hypertarget{CentorZerodimensional}
The Cantor set \( C \) is zerodimensional, i.e. it has a base of clopen sets.
\end{lemma}

\begin{thm}[Topological characterisation of the Cantor set]
\label{CantorSetUniversalProperty}
\hypertarget{CantorSetUniversalProperty}
If a topological space \( K \) is compact, metrizable, zerodimensional with no isolated points, then
\[ 
    K \cong C. 
\]
\end{thm}

\section{The group structure}

The Cantor set has a natural abelian group structure given by its product structure. We can phrase it even more efficiently when we think of \( C \) as \( \mathcal{P}( \mathbb{N} ) \) -- the symmetric difference (or xor for the informatically inclined).

\[ 
    A \oplus B := A \Delta B 
\]

Every element has order two!

\paragraph{Fact.} Together with the operation \( \oplus \), the Cantor set \( C \) is a compact topological group, i.e. the function
\[ 
    (x, y) \mapsto x \oplus y 
\]
is continuous (in general the second element is inversed, but here every element is its own inverse anyway).
