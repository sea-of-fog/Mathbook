\chapter{The Künneth formula}

This chapter accomplishes a lofty goal: to calculate the homology groups of a product topological space. 
% Along the way we will discover a new way of combining spaces together, called the \emph{smash product}.

\section{Setup}

% Doesn't this need: finite CW-complexes or of finite dimension?
We will first try and discover how this all works for CW-homologies, and then use what we've found to generalize the statements to singular homology. The reason why CW-complexes are good for the job is that given CW-structures on $X$ and $Y$, we can easily form a CW-structure on $X \times Y$. One should expect to form a simplicial structure on a product of simplicial complexes, but that would require subdividing a product of $n$- and $m$-simplices into $(n+m)$-simplices, which, if anything, is unpleasant.

We have a product structure on \( X \times Y \) given by products of cells and products of characteristic maps. Let us now try to calculate the cellular boundaries.

One of the cells is larger in dimension: then we get a coefficient of zero, because the map turns out to be constant.

If the above does not hold, then we have to reduce exactly one dimension by one. However, if in the dimension we keep we use a different cell of that dimension, the same problem applies.

Generally, we have the following:

\begin{lemma}

    Suppose we have a continuous map
    \[
    f: I^k \to I^{k-1},
    \]
    which is equal to a product map
    \[
        f = id_I \times g
    \]
    for some continuous
    \[
        g: I^{k-1} \to I^{k-2}.
    \]
    Then, the degree of
    \[
        \widetilde{f}: \Bd I^k \to I^{k-1}/\Bd I^{k-1}
    \]
    is equal to the degree of
    \[
        \widetilde{g}: \Bd I^{k-1} \to I^{k-2}/\Bd I^{k-2}.
    \]

\end{lemma}

\paragraph{Remark.} Note that since this is not the same sphere, the degree only makes sense \emph{modulo orientation}.

\section{Orientations}

Remember that when describing a CW-complex structure, there is an additional degree of freedom -- the orientation of each cell. This orientation needs to be taken into account.

%TODO: ćwiczenie Świątkowskiego

For the purposes of this chapter, it is very important to see how an orientation of the interior can be turned into an orientation of the boundary.

% TODO: add picture
One way of doing this would be to just take the \emph{vector to the right and up} in an Euclidean space. However, this leads to an inconsistent choice of orientation on the boundary. A correct way is to pick an inward facing normal at each point on the boundary and then pick an orientation on the boundary which 

This can formally be described in homology by noting that the boundary homomorphism is an isomorphism. The determined choice of generator for \( I^k \) then projects down to \( \Bd I^k \).

\section{Literature used for this chapter}

\begin{enumerate}
    \item Allen Hatcher, Algebraic Topology
    % TODO: add umlaut  
    \item MathOverflow, biography of Hermann Kunneth: https://mathoverflow.net/questions/114215/who-was-hermann-k%C3%BCnneth
    % TODO: Make this a link
    \item https://sili-math.github.io/AT2020/Lecture-22.pdf
\end{enumerate}
