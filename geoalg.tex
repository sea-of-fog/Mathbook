\chapter{Local properties in algebra}

\section{Localization of modules}

\section{Local rings}

This section is devoted to a class of rings very important in algebraic geometry --
-- local rings. Although the definition is purely algebraic, it corresponds to
notions of \textit{looking around a point} on a (affine) variety.

\subsection{Characterisations of local rings}

% TODO: add a ring subsection header
All material in this section applies to (unital) rings. They need not be commutative
for the results to work.

\begin{def}[local ring]
    A ring \( R \) having a unique maximal ideal \( \mathfrak{m} \) is called
    a {\bf local ring} and denoted \( (R, \mathfrak{m}) \).
\end{def}

It turns out there is another characterisation of such rings, which is of~particular
usefulness in working with the second point of interest of this chapter -- Discrete
Valuation Rigs, whose definition we postpone for now.

\begin{thm}[local ring characterisation]
    A ring R is local iff the set of noninvertible elements forms an ideal.
    This is the the unique maximal ideal.
\end{thm}

\begin{proof}
    The first implication is trivial, since an ideal containing a unit
    is~not proper.

    Suppose now \( R \) is local with maximal ideal \( \mathfrak{m} \).
    Since \( \mathfrak{m} \) is proper, it does not contain any invertible
    element. Take a noninvertible \( r \in R \). The principal ideal \( (r) \)
    can be extended (via Zorn's lemma) to a maximal ideal, which must be
    the unique maximal ideal \( \mathfrak{m} \), so \( r \in \mathfrak{m} \).

\end{proof}

There is also an element-wise characterisation of local rings. To see it
emerge, let us try a different line of attack for the previous proof.

We know that \( \mathfrak{m} \) contains only noninvertible elements,
but we still have to prove that it contains all such elements. By~way~of~contradiction,
suppose \( r \not\in \mathfrak{m} \). Now by maximality we have
\[ 
    \mathfrak{m} + (r) = R 
\]
and in particular
\[ 
    m + ar = 1 
\]
for some \( a \in R, m \in \mathfrak{m}\). We would like to derive a contradiction,
so we would hope that the identity above lets us conclude that \( r \) is invertible.
The property below is what we need

\begin{thm}[local ring addition]
   A ring \( R \) is local iff it has the following property: for all \( a, b \) such that  
   \[ 
      a + b \in R^\star 
  \]
    at least one of \( a,b \) is invertible. 
\end{thm}

\begin{proof}
    The line of attack above shows how this implies \( R \) being local.
    Once we have this property, we can take any maximal ideal. If there was
    some noninvertible element it did not contain, we would be able to extend it.

    On the other hand, take a local ring. We will show it is impossible to sum
    two noninvertibles to a unit. We have that
    \[ 
    (a) + (b) \subseteq \mathfrak{m} + \mathfrak{m} = \mathfrak{m}, 
   \]
   which contains no units.

\end{proof}

It should be noted that the property expressed in the previous theorem might
be rephrased in two ways: first, take an arbitrary finite sum instead of two
elements and second, have the sum be equal to \( 1 \) instead of invertible.
It should be easy to see that all such characterisations are equivalent.

\subsection{Examples and generic constructions}

\subsection{Properties of local rings}

Krull's intersection, Nakayama's, Kaplansky's theorem

\section{Valuations on Fields and Discrete Valuation Rings}

% lemma
% In a DVR, elements of valuation zero are invertible

% lemma
% In a DVR, the following are equivalent: an element is irreducible,
% an element generates the maximal ideal,

% text:
% not any ring given a valuation is a DVR! mathbb{Z} with the usual
% p-adic valuation is not a DVR (elements of valuation 0 are not)
% invertible

% in a DVR, the only ideals are (r^n) = \mathfrak{m}^n

\section{Literature used for this chapter}

\begin{enumerate}
    \item Piotr Kowalski, Algebraic Curves. Chapter 2.3.
    \item nlab page for local rings
    \item https://scholarworks.boisestate.edu/cgi/viewcontent.cgi?article=2933&context=td\
          (for Kaplansky's theorem)
    \item Wikipedia pages
    % \item Atiyah MacDonald, Introduction to Commutative Algebra, Chapters 3 and 9
\end{enumerate}

\chapter{Algebraic Geometry, problemset 5}

\subsection*{Problem 4}

Since \( v \) is a smooth point of \( C \), the ring \( \mathcal{O}_v \)
is a DVR. A local parameter is defined as any uniformizing parameter
of that DVR and having a zero of order 1 means being of valuation \( 1 \)
in that DVR.

Thus is suffices to prove the following: in a DVR, an element is of valuation
\( 1 \) iff it is irreducible.

Since a DVR is a local ring, let us denote its maximal ideal by 
\( \mathfrak{m} = (r) \). Note that \( r \) must be irreducible, otherwise
the~ideal \( \mathfrak{m} \) would not be maximal. The valuation on this DVR
can~be~described as the usual \( r \)-adic valuation or belonging to a power of 
the maximal ideal (see solution of {\bf Problem 7.}).

Let \( f \) be a local parameter, i.e. irreducible. Then, by {\bf Remark 2.37.}
\[ 
    (f) = \mathfrak{m} = (r), 
\]
so \( f \) and \( r \) are associates, therefore \( f = ur \) for some unit \( u \).
We have that \( u \in \mathcal{O}_v \setminus \mathfrak{m} \), so \( v(u) = 0 \) and
\[ 
    v(f) = v(u) + v(r) = 0 + 1 = 1. 
\]

On the other hand, let \( f \) be of valuation \( 1 \). Then, as the valuation
is the \( r \)-adic valuation we have that
\[ 
    f = ra 
\]
for some \( r \nmid a \), so \( a \in \mathcal{O}_v \setminus \mathfrak{m} \).
This implies \( a \) is a unit, so \( f \) and \( r \) are associated,
so \( f \) is irreducible since \( r \) is.

\subsection*{Problem 5}

\subsubsection*{Problem 5a}

Consider a tangent line 
\( L = V(\alpha X + \beta Y + \gamma )\).

If \( L \) is tangent, the intersection number \( I(0, L \cap C)  > 1 > 0 \),
so \( 0 \in L \), so \( \gamma = 0 \). By the same reasoning, \( F \) has zero
constant term.

It is easy to see that a line is a smooth curve. Therefore, the curve \( C \)
is tangent to \( L \) iff \( F \) is of valuation at least two in \( \mathcal{O}_0 \)
(which is a DVR). By {\bf Problem 7} that is equivalent to \( F \) being a 
member of \( \mathfrak{m}^2_v \) (where \( F \) is reinterpreted
as \( F + I(L) \) in \( K(L) \)).

Algebraically, the square of the maximal ideal corresponds to
\[ 
    I^2_L(0) \mathcal{O}_0 = \left\{ \frac{G}{H} : G \in I^2_L(0), H \not\in I_L(0)\right \} .
\]
Thus, \( F \) being of valuation at least \( 2 \) is equivalent to it being of the form
\[ 
    F = \frac{1}{N} \cdot \sum_i G_i H_i  
\]
as a rational function in \( K(L) \) for some \( N \in K[L] \setminus I_L(0)\)
and \( G_i, H_i \in I_L(0) \). This is again equivalent to the identity
\[ 
    FN = \sum_i G_i H_i 
\]
in \( K[L] \), which in turn is equivalent to \( F \) having the form
\[ 
    FN = \sum_i G_i H_i + P(\alpha X + \beta Y)
\]
for some polynomials such that \( N(0) \neq 0 \) and \( G_i (0) = H_i(0) = 0 \).

After this introduction, we will show
\[ 
    T_0 C = V\left( \partial_X F (0) X + \partial_Y F(0) Y \right) 
\]
via two inclusions, first from left to right.

If both partials are \( 0 \), the vanishing set is the whole space \( \mathbb{A}^2 \),
so this inclusion is trivial. Suppose now at least one partial is nonzero.
Differentiating both sides of the above identity w.r.t. \( X \) we get that
\[ 
    \partial_X F \cdot N + F \cdot \partial_X N = \left( \sum_i \partial_X G_i \cdot H_i
    + G_i \cdot \partial_X H_i \right ) + \partial_X P \cdot (\alpha X + \beta Y) + \alpha P.
\]
Evaluating both sides at \( 0 \) and remembering which polynomials vanish at \( 0 \)
we obtain
\[ 
    N(0) \partial_X F (0) = P(0) \alpha.
\]
We can now repeat this for differentiation w.r.t. \( Y \). This implies that the
vectors
\[ 
    [ \partial_X F(0), \partial_Y F(0) ]^T, [ \alpha, \beta ]^T \in \mathbb{K}^2
\]
are linearly dependent, so they describe the same line.

Now for the other inclusion. We will lean on the fact that the first partials
evaluated at zero are the coefficients of the monomials \( X, Y \) in the polynomial \( F \).
Suppose first both partials of \( F \) are zero.
Then we can write \( F \) as
\[ 
    F = X^2G(X,Y) + XY H(X,Y) + Y^2 P(X,Y) + 0 \cdot (\alpha X + \beta Y)
\]
which is the form we needed. This gives that \textit{any} line is tangent to \( C \)
at \( 0 \), so the tangent space is the whole plane (as is \( V(0 \cdot X + 0 \cdot Y) \)).

If, on the other hand, some partial is nonzero, we can write \( F \) as
\[ 
    F = 1 \cdot (\partial_X F (0) \cdot X  + \partial_Y F(0) Y) + X^2G(X,Y) +
    XY H(X,Y) + Y^2 P(X,Y).
\]
This proves that \( C \) is indeed tangent to \( V(\partial_X F (0) \cdot X  +
\partial_Y F(0) Y)\).

\subsubsection*{Problem 5b}

From the previous subproblem and the lecture we know that both \( T_0 C \)
and \( I_C (0) / I_C(0)^2 \) are finite-dimensional \( K \)-vector spaces.
This allows us to use a theorem of linear algebra which states that a bilinear map
such as is given in the problem induces an isomorphism iff for all \( P \neq 0 \)
the function
\[ 
    x \mapsto \Phi(x, P)
\]
is nonzero (i.e. is nonzero for some \( x \)) or, equivalently, that if this 
function is zero then so is \( P \). Here \( P \) should be understood as the
representative regular function in \( K[C] \) or a polynomial which represents
that function (so a representative of representatives).

Suppose then that this function is zero. First consider the case that both
partials of \( F \) are zero. Then the tangent space is the whole plane and
we have that for all \( x, y \in K \)
\[ 
    \partial_X P(0) x + \partial_Y P(0) y = 0,
\]
which gives that both partials of \( P \) vanish at zero, so we have the form
\[ 
    P = X^2 P_1 + XY P_2 + Y^2 P_3 
\]
for \( P \) as a polynomial, which in particular implies
\[ 
    P \in I_C(0)^2  
\]
as a regular function, so
\[ 
    P = 0 
\]
in \( I_C(0)/I_C(0)^2 \). 

Now consider what happens when \( F \) has a nonzero partial. This means that
\[ 
    \partial_X P(0) x + \partial_Y P(0) y = 0
\]
for all points such that
\[ 
    \partial_X F(0) x + \partial_Y F(0) y = 0,
\]
which gives that either the gradient of \( P \) is zero (in which case we can
repeat the reasoning from the previous case) or that the gradients of \( P \)
and \( F \) are linearly dependent. Then there exists a scalar \( \alpha \)
such that
\[ 
    P - \alpha F 
\]
has zero first partials, so
\[ 
    P = \alpha F + X^2 P_1 + XY P_2 + Y^2 P_3
\]
as a polynomial, so
\[ 
    P = X^2 P_1 + XY P_2 + Y^2 P_3 + I(C)
\]
as a regular function. As we have done many times, we now conclude that
\[ 
    P \in I_C(0)^2,
\]
which is what we needed.

\subsection*{Problem 6}

\subsubsection*{Problem 6a}

Let
\[ 
    a = r^n \frac{\alpha}{\beta}, b = r^m \frac{\gamma}{\delta},
\]
with \( r \nmid \alpha, \beta, \gamma, \delta \) and wlog \( n \geqslant m \). Then
\[ 
    a + b = r^m \frac{r^{n-m}\alpha\delta + \beta\gamma}{\beta\delta}.
\]
Note that \( r \) does not divide the numerator (since \( r \) is irreducible
and thus prime in a UFD). If \( n > m \), it will also not divide the denominator,
but if \( n = m \) it might. In the first case the valuation is exactly \( m \),
while in the second case it might change -- but only increase.

\subsubsection*{Problem 6b}

We have
\[ 
    ab = r^{n+m} \frac{\alpha\gamma}{\beta\delta}. 
\]
By virtue of \( r\) being prime, we have \( r \nmid \alpha\gamma\beta\delta \),
so
\[ 
    v_r(ab) = n + m. 
\]

\subsubsection*{Problem 6c}

For every \( n \in \mathbb{Z} \) we have
\[ 
    v_r(r^n) = n,   
\]
so the valuation is indeed surjective.

\subsection*{Problem 7}

We claim that
\[ 
    \mathfrak{m}^n = \{ x : v_R(x) \geqslant n \}, 
\]
from which the problem follows immediately. For \( n = 0 \) the claim follows
from the nonnegativity of the valuation.

Let \( r \in R \) be the uniformizing parameter. Then we have that 
\[ 
    \mathfrak{m} = (r)
\]
by {\bf remark 2.37.}, so an element of \(a \in \mathfrak{m}^n \) is of the form
\[ 
    a = \sum_i a_i r^n
\]
for some \( a_i \in R \), so 
\[ 
    v_R(a) \geqslant \min (v_r(a_1r^n), v_r(a_2r^n), \ldots, v_r(a_jr^n))
    \geqslant \min ( n, n, \ldots n ) = n.
\]
This gives us one inclusion of the claim. For the other one, take an element
\( x \) of valuation no less than \( n \). The definition of valuation implies
that for some \( m \geqslant n \)
\[ 
    x = r^m y = (r^{m-n}y) r^n \in \mathfrak{m}^n.
\]
This completes the solution.


\subsection*{Problem 8}

Take \( x, y \in \mathcal{O}_v \). Then we have
\[ 
v(xy) = v(x) + v(y) \geqslant v(y) \geqslant 0
\]
and
\[ 
    v(x + y) \geqslant \min (v(x), v(y)) \geqslant 0. 
\]
This implies that \( \mathcal{O}_v \) is a subring of \( L \).
If we take \( x,y \) such that both valuations are positive, the sum has
a positive valuation, and if at least \( y \) has a positive valuation
then the product does as well. This implies that \( \mathfrak{m}_v \) is an ideal.

Note that for any valuation we have
\[ 
    v(1) = v(1 \cdot 1) = v(1) + v(1), 
\]
so
\[ 
    v(1) = 0. 
\]
Now take any \( x \in \mathcal{O}_v \setminus \mathfrak{m}_v \). This means
that \( v(x) = 0 \). Let \( y \) be the multiplicative inverse (in \( L \)!)
of \( x \). Then
\[ 
    0 = v(1) = v(xy) = v(x) + v(y), 
\]
so \( v(y) = 0\) and \( y \in \mathcal{O}_v \). If \( v(x) > 0 \), then \( v(y) < 0\)
and \( y \not\in \mathcal{O}_v \). We have just proved that the set of noninvertible
elements of \( \mathcal{O}_v \) is an ideal. Therefore \( (\mathcal{O}_v, \mathfrak{m}_v) \)
is a local ring.

Since the valuation is surjective, \( \mathfrak{m}_v \) is nonempty and \( \mathcal{O}_v \) has
nonzero noninvertible elements, so it is not a field. To finish the proof~that
\( \mathcal{O}_v \) is a DVR all we need to do is show that \( \mathcal{O}_v \) is a PID.

To do that, take \( r \) to be any element of valuation \( 1 \). Such an element
exists by surjectivity of the valuation.

\paragraph{Claim.} Let \( t \in \mathcal{O}_v \) and \( v(t) = n \geqslant 0 \).
Then \( t = ur^n \) for some unit \( u \in \mathcal{O}_v \).

\paragraph{Proof.} Let
\[ 
    \alpha = \frac{t}{r^n} \in L. 
\]
Then \( \alpha \) has valuation \( 0 \), so it actually is an invertible
element~of \( \mathcal{O}_v \).

% \proofend

Now take an ideal \( I \). Note that if \( I \) has any element of valuation \( n \),
then by the claim above it contains all elements of valuation \( n \) as they all are
associated with \( r^n \). It will also contain an element (so all elements) of higher
valuations by virtue of being closed under multiplication by \( r \).

This lets us conclude that 
\[ 
    I = (r^n) 
\]
where \( n \) is the smallest valuation achievable by an element of \( I \).
