\chapter{Logika R, Problemset 1}

\section*{Zadanie 9}

\begin{proof}[\( (\Leftarrow) \)]
    Skoro \( X \vdash \varphi \to \psi \), to tym bardziej \( X \cup \{\varphi\} \vdash \varphi \to \psi \). Niech \( \alpha_1, \alpha_2, \ldots, \alpha_n = \varphi \to \psi \) będzie dowodem świadczącym, że \( X \cup \{ \varphi \} \vdash \varphi \to \psi \). Korzystając z reguły \textbf{MP} możemy rozszerzyć ten dowód do dowodu formalnego \( \alpha_1, \alpha_2, \ldots, \alpha_n = \varphi \to \psi, \psi \), więc \( X \cup \{\varphi\} \vdash \psi \).
\end{proof}

\begin{proof}[\( (\Rightarrow) \)]
    Niech \( \alpha_1, \alpha_2, \ldots, \alpha_n = \psi \) będzie dowodem formalnym poświadczającym, \( X \cup \left\{ \varphi \right\} \vdash \psi \). Pokażemy, że z \( X \) można udowodnić \( \varphi \to \psi \) poprzez dowód \( \varphi \to \alpha_1, \varphi \to \alpha_2, \ldots, \varphi \to \psi \).

    \paragraph{Claim.} Dla każdego \( 0 < i \leqslant n \), \( X \cup \left\{ \varphi \right\} \vdash \alpha_i \).

\begin{proof}[Dowód claimu]
    Dowód \textbf{claimu} prowadzimy przez silną indukcję względem \( i \). Rozważamy przypadki w zależności od tego, jak \( \alpha_i \) zostało wprowadzone do dowodu \( X \cup \left\{ \varphi \right\} \vdash \psi\).

    \begin{enumerate}
        \item \textbf{Przesłanka \( \varphi \).} Ponieważ \( v \to v \) jest tautologią KRZ, możemy skorzystać z aksjomatu \textbf{A0} i pod tę tautologię podstawić \(v := \varphi \), otrzymując \( \varphi \to \varphi \) bez konieczności korzystania z \( \varphi \) jako przesłanki.
        \item Przesłanka \( \eta \in X \). Ponieważ wyprowadzamy dowód z \( X \), możemy użyć tej samej przesłanki. Potem korzystamy z tautologii KKRZ \( v \to (w \to v) \), podstawiając w niej \( v := \eta \), \( w := \varphi \) i korzystamy z reguły \textbf{MP}, by otrzymać dowód \( \varphi \to \eta \).
        \item Aksjomat. Zauważmy, że dalej możemy uzyskać \( \alpha_i \) jako aksjomat. Ponadto, schemat \( v \to (w \to v) \) jest tautologią KRZ, możemy więc jako przykład tautologii uzyskać formułę \( \alpha_i \to (\varphi \to \alpha_i) \), po czym skorzystać z reguły \textbf{MP}, by uzyskać dowód \( \varphi \to \alpha_i \).
        \item Modus ponens. Załóżmy, że \( \alpha_i \) wyprowadziliśmy z \( \alpha_j \) i \( \alpha_k = \alpha_j \to \alpha_i \). Z założenia indukcyjnego \( X \vdash \varphi \to \alpha_j \) i \( X \vdash \varphi \to (\alpha_j \to \alpha_i) \). Schemat \( (v \to w) \to ((v \to (w \to u)) \to (v \to u)) \) jest tautologią KRZ, więc możemy jako jego przykład uzyskać \( (\varphi \to \alpha_j) \to (\varphi \to (\alpha_j \to \alpha_i) \to (\varphi \to \alpha_i)) \), a potem zasotoswać dwukrotnie regułę \textbf{MP}.
        \item \( \forall \)-reguła. Załóżmy, że \( \alpha_i \) wyprowadziliśmy z \( \alpha_j \) za pomocą \( \forall \)-reguły. Wtedy, zzałożenia indukcyjnego \( X \vdash \varphi \to \alpha_j \). Korzystając z \( \forall \)-reguły, możemy udowodnić \( X \vdash \forall v\,\,\varphi \to \alpha_j \), a potem z aksjomatu \textbf{A1} (\( \varphi \) jest zdaniem, więc nie ma zmiennych wolnych) oraz reguły \textbf{MP} uzyskać \( \varphi \to \forall v\,\,\alpha_j \equiv \varphi \to \alpha_i \).
    \end{enumerate}
    
\end{proof}

\end{proof}

\section*{Zadanie 10}

\begin{proof}[\((\Leftarrow)\)]
    Jeśli \( \mathrm{Cn(X)} = \mathrm{Form}_L \), to \( X \vdash \varphi \wedge \neg\varphi \) dla dowolnego zdania \( \varphi \in \mathrm{Prop}_L \), w szczególności np. dla zdania \( \varphi = \forall x. x = x \). Zatem \( X \) jest sprzeczny.
\end{proof}

\begin{proof}[\( (\Rightarrow) \)]
Załóżmy, że \( X \vdash \varphi \wedge \neg \varphi \) i rozważmy dowolną formułę \( \psi \in \mathrm{Form}_L \).
Zauważmy, że \( \alpha \wedge \neg \alpha \to \beta \) jest tautologią KRZ. Korzystając z aksjomatu \textbf{A0} oraz podstawiając \( \varphi, \psi \) pod odpowiednio \( \alpha, \beta \) otrzymujemy
\[ 
    X \vdash \varphi \wedge \neg \varphi \to \psi.
\]
Ponieważ \( X \vdash \varphi \wedge \neg \varphi \), możemy dowód formalny powyższego osądu rozszerzyć, korzystając z reguły \textbf{MP}, do dowodu
\[ 
    X \vdash \psi. 
\]
Ponieważ \( \psi \) było wybrane dowolnie, \( \mathrm{Cn}(X) = \mathrm{Form}_L \)
\end{proof}

\section*{Zadanie 11}

Skonstruujemy szukane rozszerzenie korzystają z \textbf{lematu Kuratowskiego-Zorna}. 

\paragraph{Claim 1.} Niech \( X \) będzie niezupełną, niesprzeczną teorią w języku \( L \), w szczególności niech \( \varphi \) będzie zdaniem takim, że \( X \not\vdash \varphi \) oraz \( X \not\vdash \neg\varphi \). Wtedy \( \mathrm{Cn}(X \cup \left\{ \varphi \right\}) \) jest niesprzeczną teorią ściśle większą niż \( X \).

\begin{proof}[Dowód claimu 1]
Załóżmy nie wprost, że \( X \cup \left\{ \varphi \right\} \) jest sprzeczne. Istnieje zatem zdanie \( \alpha \in \mathrm{Prop}_L \) t. że \( X \cup \left\{ \varphi \right\} \vdash \psi \wedge \neg \psi \). Ponieważ \( (\alpha \wedge \neg \alpha) \to \beta \) jest tautologią KRZ, możemy skorzystać z aksjomatu \textbf{A0} oraz reguły \textbf{MP} możemy udowodnić (podstwiając \( \psi \) za \( \alpha \) oraz \( \neg\varphi \) za \( \beta \)), że \( X \cup \left\{ \varphi \right\} \vdash \neg\varphi\). Z tw. o dedukcji wnioskujemy zatem, że \( X \vdash \varphi \to \neg\varphi \). Ponieważ \( (\alpha \to \neg\alpha) \to \neg\alpha \) jest tautologią KRZ, korzystając z \textbf{A0} oraz \textbf{MP}, że \( X \vdash \neg\varphi \), co jest sprzeczne z założeniami. 
\end{proof}


\paragraph{Claim 2.} Niech \( \left\{ X_\alpha \right\} \) będzie wstępujących łańcuchem niesprzecznych teorii języka \( L \). Wtedy suma \( \bigcup_\alpha X_\alpha \) też jest niesprzeczną teorią.

\begin{proof}[Dowód claimu 2.]
    Claim 2. wynika z finitarnego charakteru dowodliwości. Dokładniej, załóżmy że \( \bigcup_\alpha X_\alpha \vdash \psi \wedge \neg \psi \) dla pewnego zdania \( \psi \in \mathrm{Prop}_L \). W dowodzie formalnym użytych jest tylko skończenie wiele przesłanek.

    Nazwijmy je \( \varphi_1, \varphi_2, \ldots, \varphi_n \). Każda z nich pochodzi z którejś teorii \( X_\alpha \), powiedzmy zatem, że \( \varphi_k \in X_k \). Wtedy w szczególności któraś z teorii \( X_1, X_2, \ldots, X_n \) jest największa, powiedzmy \( X_k \), więc wszystkie przesłanki \( \varphi_i \) do niej należą. To jednak oznaczałoby, że \( X_k \) jest sprzeczna, co prowadzi do sprzeczności.

    Podobnie dowodzimy, że wstępująca suma jest teorią.
\end{proof}

Rozważmy zbiór teorii języka \( L \) rozszerzających \( X \), częściowo uporządkowany przez inkluzję. Z \textbf{claimu 2.} każdy łańcuch w tym zbiorze uporządkowanym ma ograniczenie górne (sumę teorii z tego łańcucha), z \textbf{lematu Kuratowskiego-Zorna} istnieje zatem element maksymalny \( X' \). Z \textbf{claimu 1.} wynika, że dla każdego zdania \( \varphi \) zachodzi albo \( X' \vdash \varphi \) albo \( X' \vdash \neg\varphi \), inaczej moglibyśmy poszerzyć \( X' \) do jeszcze większej teorii rozszerzającej X.
