\chapter{Munkres}

\section{The Countability Axioms}

\subsection{Definitions}

\begin{defn}[Basis at a point]
\label{BasisAtPoint}
\hypertarget{BasisAtPoint}
Let \( x \in X \). A collection \( \mathcal{B} \) of neighbourhoods of \( x \) is a \textbf{basis at \( x \)} iff every open neighbourhood \( U \ni x \) contains a member of \( \mathcal{B} \), i.e. there is a \( B \in \mathcal{B} \) s. that \( x \in B \subseteq U \).
\end{defn}

\begin{defn}[First countability]
\label{FirstCountable}
\hypertarget{FirstCountable}
A space \( X \) is said to be \textbf{first-countable} if it has a countable basis \( \mathcal{B}_x \) at each of its points \( x \in X \).
\end{defn}

\begin{defn}[Second countability]
\label{SecondCountable}
\hypertarget{SecondCountable}
A space \( X \) is said to be \textbf{second-countable} if it has a countable basis.
\end{defn}

\begin{defn}[Dense subset]
\label{DenseSubset}
\hypertarget{DenseSubset}
A subset \( D \subseteq X \) is said to be \textbf{dense} if
\[ 
    \Cl D = X. 
\]
\end{defn}

\begin{defn}[Lindel\"of]
\label{Lindelof}
\hypertarget{Lindelof}
A space \( X \) is called \textbf{Lindel\"of} if every open cover of \( X \) admits a countable subcover.
\end{defn}

\subsection{Theorems}

\subsection{Chapter review questions}

\begin{enumerate}
    \item What is the significance of first-countability for closures?
    \item What is the significance of first-countability for continuous function?
    \item What examples does Munkres give for a first-countable space that is not second-countable?
    \item How large can a discrete subspace of a second-countable space be?
    
    \textbf{Answer:} At most countable. A discrete subspace can be injected into the basis, so it has a smaller size.
    \item What are the relations between countability propeties?
    \item What is the significance of the Sorgenfrey line \( \mathbb{S} \) for countability axioms? Which countability properties does it have?
    \item What is the significance of the Sorgenfrey plane \( \mathbb{S}^2 \) for countability axioms? Which countability properties does it have?
    \item What is the significance of the ordered square \( I_o^2 \) for countability axioms? Which countability properties does it have?
    \item How can you make it easier to check the Lindel\"of property?
    \item Under what conditions do the relations between countability axioms reverse, i.e. what can you use to conclude that a space is second-countable from the other axioms?
\end{enumerate}

\subsection{Questions about property preservation}
\begin{enumerate}
    \item Is first-countability hereditary?
    \item Is first-countability a productive property?
    \item Is first-countability a coproductive property?
    \item Is first-countability a surjective propety?
    \item is first-countability preserved under quotients?
    \item Is second-countability hereditary?
    \item Is second-countability a productive property?
    \item Is second-countability a coproductive property?
    \item Is second-countability a surjective propety?
    \item is second-countability preserved under quotients?
    \item Is being Lindel\"of hereditary?
    \item Is being Lindel\"of a productive property?
    \item Is being Lindel\"of a coproductive property?
    \item Is being Lindel\"of a surjective propety?
    \item is being Lindel\"of preserved under quotients?
    \item Is separability hereditary?
    \item Is separability a productive property?
    \item Is separability a coproductive property?
    \item Is separability a surjective propety?
    \item Is separability preserved under quotients?
\end{enumerate}
