\chapter{Munkres}

\section{The Countability Axioms}

\subsection{Definitions}

\begin{defn}[Basis at a point]
\label{BasisAtPoint}
\hypertarget{BasisAtPoint}
Let \( x \in X \). A collection \( \mathcal{B} \) of neighbourhoods of \( x \) is a \textbf{basis at \( x \)} iff every open neighbourhood \( U \ni x \) contains a member of \( \mathcal{B} \), i.e. there is a \( B \in \mathcal{B} \) s. that \( x \in B \subseteq U \).
\end{defn}

\begin{defn}[First countability]
\label{FirstCountable}
\hypertarget{FirstCountable}
A space \( X \) is said to be \textbf{first-countable} if it has a countable basis \( \mathcal{B}_x \) at each of its points \( x \in X \).
\end{defn}

\begin{defn}[Second countability]
\label{SecondCountable}
\hypertarget{SecondCountable}
A space \( X \) is said to be \textbf{second-countable} if it has a countable basis.
\end{defn}

\begin{defn}[Dense subset]
\label{DenseSubset}
\hypertarget{DenseSubset}
A subset \( D \subseteq X \) is said to be \textbf{dense} if
\[ 
    \Cl D = X. 
\]
\end{defn}

\begin{defn}[Lindel\"of property]
\label{Lindelof}
\hypertarget{Lindelof}
A space \( X \) is called \textbf{Lindel\"of} if every open cover of \( X \) admits a countable subcover.
\end{defn}
.
\subsection{Lemmas}

\begin{lemma}[Picking a decreasing basis at point]
\label{PointBasisDecreasing}
\hypertarget{PointBasisDecreasing}
Let \( \mathcal{B} = \left\{ B_1, B_2, \ldots \right\} \) be a basis for the topology of \( X \) at a point \( x \in X \). Then the sequence
\[ 
    B_1, B_1 \cap B_2, B_1 \cap B_2 \cap B_3, \ldots 
\]
is a decreasing inscribed basis at \( X \).
\end{lemma}

\begin{proof}
    All of these sets contain \( x \). Since any neighbourhood of \( U \ni x \) contains some \( B_k \), it will also contain the set \( B_1 \cap B_2 \cap \ldots \cap B_k \).
\end{proof}

\begin{lemma}[Local basis in a topological group]
\label{TopGroupLocalBasis}
\hypertarget{TopGroupLocalBasis}
Let \( \mathcal{B} = \left\{ B_1, B_2, \ldots \right\} \) be a basis for a first-countable topological group \( G \) at the identity \( e \). Then, the basis sets \( B_k \) can be shrunk to sets \( C_k \) such that:
\begin{enumerate}
    \item \( C_k \subseteq B_k \),
    \item \( C_k^{-1} = C_k \),
    \item \( C_{k+1} \cdot C_{k+1} \subseteq C_{k-1} \),
    \item \( e \in C_k \).
\end{enumerate}
In particular, the family \( \mathcal{C} = \left\{ C_1, C_2, \ldots \right\} \) is a decreasing basis for the topology of \( G \) at \( e \). 
\end{lemma}

\begin{proof}
    The first trick we'll use is that since the inverse operation \( \iota \) is a homeomorphism and its own inverse, the set 
    \[ 
        U \cap U^{-1} 
   \]
   is open, contains \( e \) and is closed under taking inverses. So, for \( C_1 \) we pick \( B_1 \cap B_1^{-1} \). Now suppose we have already chosen \( C_1, C_2, \ldots, C_k \). Since the group operation is continuous on the product, we can choose an open set \( U \ni e \) such that \( U \cdot U \subseteq C_k \). Now the set
   \[ 
       U \cap B_{k+1} 
  \]
  satisfies all the conditions apart from (2), so we perform on it the inverse trick mentioned at the beginning. Explicitly, we set
  \[ 
      C_{k+1} = (U \cap B_{k+1}) \cap (U \cap B_{k+1})^{-1}.
 \]
\end{proof}

\begin{lemma}[Lindel\"of by basis]
\label{LindelofByBasis}
\hypertarget{LindelofByBasis}
A topological space \( X \) is Lindel\"of iff a countable cover can be chosed from any cover by elements of a given basis \( \mathcal{B} \).
\end{lemma}

\begin{proof}
    In one direction, a basis is a cover, so the Lindel\"of basis lets one pick a countable subcover.

    In the other direction, suppose \( \mathcal{U} \) is a cover. Then, for each \( x \in X \) pick a \( U_x \in \mathcal{U} \) and basis element \( B_x \in \mathcal{B} \) such that \( x \in B_x \subseteq U_x \). The chosen basis elements form a cover by basis elements, and if we extract a countable subcover by \( B_{x_n} \), we can also extract a countable subcoer from the corresponding \( U_{x_n} \). 
\end{proof}

\begin{lemma}[Lindel\"of property by inscribed cover]
\label{LindelofByInscription}
\hypertarget{LindelofByInscription}
Fix an open covering \( \mathcal{V} \) of a topological space \( X \). Then \( X \) has the Lindel\"of property iff every cover \( \mathcal{U} \) inscribed in \( \mathcal{V} \) has a countable subcover.
\end{lemma}

\begin{proof}
    Only the backward implication is nontrivial. Suppose \( \mathcal{V} = \left\{ V_\alpha \right\} \) and \( \mathcal{U} = \left\{ U_{\beta} \right\} \) is an arbitrary cover. Then we can form an \emph{intersection covering} by
    \[ 
       \mathcal{W} := \left\{ V_\alpha \cap U_\beta \right\}.
   \]
   The cover \( \mathcal{W} \) is definitely open and inscribed in \( \mathcal{V} \), so we may choose a countable subcover from it. But if the subcover is \( V_{\alpha_n} \cap U_{\beta_n} \), then \( U_{\beta_n} \) also is a cover, so \( \mathcal{U} \) has a countable subcover.
\end{proof}

\begin{lemma}[Lindel\"of property of the Sorgenfrey line]
\label{SorgenfreyLineLindelof}
\hypertarget{SorgenfreyLineLindelof}
The Sorgenfrey line \( \mathbb{S} \) has the Lindel\"of property.
\end{lemma}

\begin{proof}
    Using \ref{LindelofByBasis} we reduce to the case of a cover by basis elements \( [a_\alpha, b_\alpha) \). Since it is a cover, we have
    \[ 
        \mathbb{R} = \bigcup_{\alpha} [a_\alpha, b_\alpha).
   \]
   We will that the set
   \[ 
       C := \mathbb{R} \setminus \bigcup_{\alpha} (a_\alpha, b_\alpha) 
  \]
  is countable, which will finish the proof, because we will then be able to pick one element from the original cover for each element of the difference and the fact that \( \mathbb{R} \setminus C \) in the Euclidean topology is second countable by \ref{FirstSecondCountabilityHereditary}, and so also Lindel\"of.

  Let \( c \in C \). Then, \( c = a_\alpha \) for some \( \alpha \). Choosing a rational number \( q \in [a_\alpha, b_\alpha) \) we get a function \( f: C \to \mathbb{Q} \). This is an injection, since if \( c_1 \neq c_2 \), then without loss of generality \( c_1 < c_2 \). The cover intervals with left endpoints \( c_1, c_2 \) have to be disjoint, so \( f(c_1) < f(c_2) \) and \( f \) is in particular injective.
\end{proof}

\begin{lemma}[Sorgenfrey line is not second-countable]
\label{SorgenfreyLineNotSecondCountable}
\hypertarget{SorgenfreyLineNotSecondCountable}
The Sorgenfrey line \( \mathbb{S} \) is not second-countable.
\end{lemma}

\begin{proof}
    Let \( \mathcal{B} \) be a basis. Then, for each \( s \in \mathbb{S} \), the open set \( [x,\,x+1) \) is the sum of some basis elements. In particular, among them there is a \( B_x \in \mathcal{B} \) which contains \( x \). Since \( \inf B_x = x \), the assignment \( x \mapsto B_x \) is injective, so \( \left| \mathcal{B} \right| \geqslant \left| \mathbb{S} \right| = \mathfrak{c} \).
\end{proof}

\paragraph{Remark.} In general, in a second-countable space, any basis contains a countable basis.

\subsection{Theorems}

\begin{thm}[Continuity in first-countable spaces]
\label{FirstCountableContinuousFunctions}
\hypertarget{FirstCountableContinuousFunctions}
If \( f: X \to Y \) is continuous, then for all sequences \( x_n \to x \) we have \( f(x_n) \to f(x) \). If additionally \( X \) is first-countable, then then converse is also true. 
\end{thm}

\begin{proof} 
    For the forward implication, pick any \( U \in N_{f(x)} \). Then \( V := f^{-1}[U] \) is open, so, for all but finitely many indices \( n \), \( x_n \in V \). Then for the same indices \( f(x_n) \in U \).

    For the backward, let \( B_n \) be a countable basis of X at \( x \) and suppose without loss of generality that \( B_n \) is decreasing. Pick again an open neighbourhood \( U \in N_{f(x)} \). If \( f^{-1}[U] \) is not open, then choose a sequence \( x_n \in B_n \setminus f^{-1}[U] \). This means that \( f(x_n) \not\in U \) for all \( n \), but \( f(x_n) \to f(x) \), a contradiction.
\end{proof}

\begin{thm}[Closures in first-countable spaces]
\label{FirstCountableClosures}
\hypertarget{FirstCountableClosures}
If \( x_n \in A \) and \( x_n \to x \), we can conclude that \( x \in \Cl A \). The converse holds in a first-countable space.
\end{thm}

\begin{proof}
    For the forward implication, every neighbourhood \( U \ni x \) contains a point of \( A \), and even infinitely many of them, so \( x \in \Cl A \).

    For the backward implication, suppose \( x \in \Cl A \) and \( \mathcal{B} = \left\{ B_1, B_2, \ldots \right\} \) is a countable basis at \( x \), \hyperlink{PointBasisDecreasing}{without loss of generality decreasing}. Then every set \( B_n \) contains a point of \( A \), so we can for a sequence \( x_n \in A \cap B_n \). Since \( B_i \searrow \), for any neighbourhood \( U \ni x \), \( B_N \subseteq U \) for some \( N \), and so \( U \) contains all \( x_n \) for \( n \geqslant N \). Therefore \( x_n \to x \) is a sequence of elements of \( A \) convergent to \( x \).
\end{proof}

\begin{thm}[First- and second-countability are hereditary]
\label{FirstSecondCountabilityHereditary}
\hypertarget{FirstSecondCountabilityHereditary}
Let \( X \) be second- or first-countable space. Then, if \( Y \) is subspace of \( X \), \( Y \) is also, respectively, second- or first-countable.
\end{thm}

\begin{proof}
    Let \( B_k \) be a basis for \( X \). Then \( B_k \cap Y \) is a basis for \( Y \). Indeed, any open set \( U \subseteq Y \) is of the form \( V \cap Y \) for some open \( V \subseteq X \). Then \( V \supseteq B_k \) for some \( k \), to \( U = V \cap Y \supseteq B_k \cap Y \). An analogous argument holds for a basis at \( y \in Y \).
\end{proof}

\subsection{Problems}

\begin{problem}[P30.9 from \cite{Mun14}]
    \label{MunkresP30.9}
    \hypertarget{MunkresP30.9}
    Let \( F \subseteq X \) be a closed subspace. Show that if \( X \) is Lindel\"of, then so is \( F \). Show also that even if \( X \)is separable, \( F \) need not be separable.
\end{problem}

\begin{proof}[Lindel\"of]
    Take a cover \( \mathcal{U} = \left\{ U_\alpha \right\} \) of \( F \) in the subspace topology. Then \( U_\alpha = V_\alpha \cap F \) for some \( V_\alpha \) open in \( X \). The family \( \left\{ F^c \right\} \cup \left\{ V_\alpha \right\} \) forms an open cover of \( X \), so by the Lindel\"of axiom we may extract a countable subcover \( \left\{ F^c \right\} \cup \left\{ V_n \right\} \). Then, \( F \subseteq \bigcup_n V_n \), so the corresponding \( U_n \) form a countable subcover of \( \mathcal{U} \) in the subspace \( F \).
\end{proof}

\begin{proof}[Separability]
The Sorgenfrey \( \mathbb{S}^2 \) is separable with dense set \( \mathbb{Q}^2 \). Consider the \emph{antidiagonal}, i.e. the subspace given by
\[ 
    \Delta^- := \left\{ (x, -x) : x \in \mathbb{R} \right\}.
\]
Since \( \Delta^- \cap [x,\, x+1) \times [-x,\, -x + 1) \cap \left\{ (x,-x) \right\} \) for all \( x \in \mathbb{R} \), the subspace \( \Delta^- \) is discrete uncountable, so it cannot have a countable dense set by a \hyperlink{MunkresP30.13}{Munkres problem}.
\end{proof}

\begin{problem}[P30.11 from \cite{Mun14}]
\label{MunkresP30.11}
\hypertarget{MunkresP30.11}
    Let \( f: X \to Y \) be continuous. Show that if \( X \) is Lindel\"of, or if \( X \) has a countable dense subset, then \( f[X] \) satisfies the same condition.
\end{problem}

\begin{proof}[Lindel\"of]
    Let \( \mathcal{U} = \left\{ U_\alpha \right\} \) be an open cover of \( f[X] \) (in the subspace topology). Then, since preimges preserve set-theoretic operations, \( f^{-1} \mathcal{U} := \left\{ f^{-1}[U_\alpha] \right\} \) is an open over of \( X \), which has a countable subcover \( f^{-1}[U_n] \) since \( X \) is Lindel\"of. As preimages preserve set-theoretic operations, \( U_n \) is a subcover of \( \mathcal{U} \).
\end{proof}

\begin{proof}[Separability]
    Let \( D \subseteq X\) be a countable dense subset and choose an arbitrary open set \( U \subseteq f[X] \). Since \( D \) is dense, there is an element \( d \in D \cap f^{-1}[U] \), so \( f(d) \in U \). This proves that \( f[D] \) is a countable dense subset of \( f[X] \).
\end{proof}

\begin{problem}[P30.12 from \cite{Mun14}]
    \label{MunkresP30.12}
    \hypertarget{MunkresP30.12}
    Let \( f: X \to Y \) be continuous and open. If \( X \) is first- or second-countable, then \( f[X] \) is respectively first- or second-countable.
\end{problem}

\begin{proof}
    We prove that if \( f \) is continous and open and \( \mathcal{B} = \left\{ B_\alpha \right\} \) is a basis of \( X \), then \( \mathcal{C} = \left\{ f[B_\alpha] \right\} \) is a basis of \( f[X] \).

    First of all, since \( f \) is open, every set \( f[B_\alpha] \) is open in \( f[X] \). Choose now an arbitrary \( y \in f[X] \) and an open neighbourhood \( U \ni y \). Then there is an \( x \in X \) wand a basis element \( x \in B_\alpha \subseteq f^{-1}[U] \), so \( y \in f[B_\alpha] \subseteq U \), which is what we need from a basis.

    For first-countability, there is an analogous claim (with an analogous proof): if \( B_n \ni x \) is a basis at some \( x \in X \), then \( f[B_n] \) is a basis at \( f(x) \).
\end{proof}

\begin{problem}[P30.13 from \cite{Mun14}]
    \label{MunkresP30.13}
    \hypertarget{MunkresP30.13}
    Show that if \( X \) is separable, then every collection of disjoint open sets in \( X \) is at most countable.    
\end{problem}
\begin{proof}
    Let \( \mathcal{U} = \left\{ U_\alpha \right\} \) be the collection of disjoint open sets and \( D \) a countable dense set. Then, every \( U_\alpha \) contains a \( d \in D \). Since the sets \( U_\alpha \) are disjoint, then the function \( U_\alpha \mapsto d \in U_\alpha \) is injective, so the cardinality of \( \mathcal{U} \) is at most \( \left| D \right| \leqslant \aleph_0 \).
\end{proof}

\begin{problem}[P30.18 from \cite{Mun14}]
\label{MunkresP30.18}
\hypertarget{MunkresP30.18}
    Let \( G \) be a first-countable topological group. Show that if \( G \) is either separable or Lindel\"of, then \( G \) is also second-countable.
\end{problem}

\begin{proof}[Solution (for separable spaces)]
    Let \( D = \left\{ d_1, d_2, \ldots \right\} \) be the countable dense subset and \( \mathcal{B} = \left\{ B_1, B_2 \ldots \right\} \) a basis at \( e \). Suppose without loss of generality, that \( \mathcal{B} \) satisfies the hypothesis of \hyperlink{TopGroupLocalBasis}{the topological group basis lemma}. We will use the dense subset to \emph{spread} the local basis over the group. Concretely, we claim that the countable basis can be given by
    \[ 
       d_m B_n,
   \]
   i.e. left translations of the basis at \( e \). Pick any open set \( U \subseteq G \) and its element \( g \in U \subseteq G \). The left-transalted set \( g^{-1}U \) contains \( e = g^{-1}g \), so we can pick a neighbourhood \( B_n \subseteq g^{-1}U \) or, equivalently, \( g \in gB_n \subseteq U \).

   At this point it might not be true that for some \( d_m \) we will have \( g \in d_mB_n \subseteq U \). However, we can shrink \( B_n \) to \( B_{n+1} \) and get that \( B_{n+1} \cdot B_{n+1} \subseteq B_m \). Then, pick a \( d_m \in gB_{n+1} \). We can now compute that
\begin{align*}
    d_m &\in gB_{n+1}  \\
    g^{-1}d_m &\in B_{n+1} \\
    e \in g^{-1}d_mB_{n+1} &\subseteq B_{n+1}B_{n+1}\subseteq B_n \\
    g \in d_mB_{n+1} &\subseteq gB_n \subseteq U, \\
\end{align*}
which completes the proof. The identity belongs to \( g^{-1}d_mB_{n+1} \) because \( g^{-1}d_m \in B_{n+1} \) and \( B_{n+1} \) is closed under inverses.
\end{proof}

\begin{proof}[Solution (for Lindel\"of spaces)]
    Let \( \mathcal{B} = \left\{ B_1, B_2, \ldots \right\} \) be a countable basis at \( e \) and again suppose it satisfies the properties of \hyperlink{TopGroupLocalBasis}{a nice topological group basis}. For a given \( B_k \), consider the cover
    \[ 
       \mathcal{A}_k = \left\{ gB_k : g \in G \right\}.
   \]
   By the Lindel\"of property, we can pick a countable subset \( G_k \subseteq G \) such that the family \( \left\{ gB_k : g \in G_k \right\} \)is a subcover of \( \mathcal{A}_k \). Now do this for all \( k \in \mathbb{N} \) and, for convenince, take
   \[ 
       %TODO: upward sum
       G_0 := \bigcup_{k=1}^\infty G_k. 
  \]
  We claim that sets of the form
  \[ 
     g_0B_k 
 \]
 for \( g_0 \in G_0 \) and \( B_k \in \mathcal{B} \) form a countable basis. Pick any open set and its element \( g \in U \subseteq G \). We repeat the solution of the previous subproblem. To be explicit, we can find an \( n \) such that \( gB_n \subseteq U \). Now \( B_{n+1} \cdot B_{n+1} \subseteq B_{n} \). Since the family \( \left\{ gB_{n+1} : g \in G_0 \right\} \) is a cover, we can find some \( g_0 \) s. that \( g \in g_0B_{n+1} \subseteq gB_n \subseteq U \).
\end{proof}

\paragraph{Remark.} The set \( G_0 \) is actually a dense set for the topology of \( G \).

\paragraph{Remark.} This is one of the situations in which the relationships between countability axioms can be reversed, that is a subset of weaker axioms implies second-countability.

\subsection{Chapter review questions}

\begin{enumerate}
    \item What is the significance of first-countability for closures?

    \textbf{Answer:} In a first-countable space closures are determined by sequences. That is, \( x \in \Cl A \) iff there is a sequence \( x_n \in A \) s. that \( x_n \to x \). In general only the backward implication holds. See \hyperlink{FirstCountableClosures}{here} for a proof.

    \item What is the significance of first-countability for continuous function?

    \textbf{Answer:} If \( X \) is first-countable, then a function \( f: X \to Y \) is continuous iff for all sequences \( x_n \to x \) we have \( f(x_n) \to f(x) \). In general, only the forward implication holds. Prove \hyperlink{FirstCountableContinuousFunctions}{this}.
    
    \item What examples does Munkres give for a first-countable space that is not second-countable?

        \textbf{Answer:} \( \mathbb{R}^\omega \) is first-countable as it is a metric space, but is not second-countable, because it has a discrete set \( \{ 0,\,1 \}^\omega \) of size \( \mathfrak{c} \).
    \item How large can a discrete subspace of a second-countable space be?
    
    \textbf{Answer:} At most countable. A discrete subspace can be injected into the basis, so it has a smaller size.
    \item What are the relations between countability propeties?

    \textbf{Answer:} Second-countability implies all other properties. See \hyperlink{MetrisableSpaceCountabilityProperties}{here} for a proof. Not even a conjuction of the three others implies second-countability, as shown by the Sorgenfrey line \( \mathbb{S} \).
    \item What is the significance of the Sorgenfrey line \( \mathbb{S} \) for countability axioms? Which countability properties does it have?

        \textbf{Answer: } The Sorgenfrey line has all three weaker countability properties, but is not second countable (\hyperlink{SorgenfreyLineNotSecondCountable}{proof}), establishing that even a conjunction of all three weaker axioms does not give second-coutnability. It is separable with dense set \( \mathbb{Q} \), first countable by choice of \( [x, x + \frac{1}{n}) \) at each \( x \), and Lindel\"of. For the last one see \hyperlink{SorgenfreyLineLindelof}{here}.

    \item What is the significance of the Sorgenfrey plane \( \mathbb{S}^2 \) for countability axioms? Which countability properties does it have?

        \textbf{Answer:}  The Sorgenfrey plane is first-countable, separable, but neither second-countable nor Lindel\"of. Thus it is a product of Lindel\"of spaces which is not Lindel\"of, establishing that Lindel\"ofness is not productive. The proof uses the fact that the antidiagonal is a closed discrete subspace.

    \item What is the significance of the ordered square \( I_o^2 \) for countability axioms? Which countability properties does it have?
    \item How can you make it easier to check the Lindel\"of property?

        \textbf{Answer: } You can restrict your attention to \hyperlink{LindelofByBasis}{covers by basis elements}, or to \hyperlink{LindelofByInscription}{covers inscribed in an arbitrary cover}.
    \item Under what conditions do the relations between countability axioms reverse, i.e. what can you use to conclude that a space is second-countable from the other axioms?
    
        \textbf{Answer:} (1) metrisable spaces are all first-countable, but either the Lindel\"of property or separability imply second-countability (\hyperlink{MetrisableSpaceCountabilityProperties}{proof}) (2) a first-countable topological group satisfying either the Lindel\"of property or separability is second-countable (\hyperlink{MunkresP30.18}{proof}).
\end{enumerate}

\subsection{Questions about property preservation}
\begin{enumerate}
    \item How does first-countability behave under taking subspaces?

    \textbf{Answer: } First-countability is preserved under taking subspaces. See \hyperlink{FirstSecondCountabilityHereditary}{the relevant theorem}.
    \item How does first-countability behave under products?
    \item How does first-countability behave under coproducts?

    \textbf{Answer: } Since first-countability only concerns what happens around a point, it is preserved under arbitrary coproducts.
    \item How does first-countability behave under surjections?

    \textbf{Answer: } First countability is preserved under open, continuous surjections. See \hyperlink{MunkresP30.12}{the relevant Munkres problem}.
    \item How does first-countability behave under quotients?
    \item How does second-countability behave under taking subspaces?
    \textbf{Answer: } Second-countability is preserved under taking subspaces. See \hyperlink{FirstSecondCountabilityHereditary}{here}.
    \item How does second-countability behave under products?
    \item How does second-countability behave under coproducts?

    \textbf{Answer: } Second countability is preserved under countable coproducts, but not larger nontrivial coproducts.
    \item How does second-countability behave under surjections?
    \item How does second-countability behave under quotients?
    \item How does being Lindel\"of behave under taking subspaces?
    \item How does being Lindel\"of behave under products?
    \item How does being Lindel\"of behave under coproducts?

    \textbf{Answer: } The Lindel\"of property is preserved under only countable coproducts if the summands are nontrivial, i.e. nonempty.
    \item How does being Lindel\"of behave under surjections?

    \textbf{Answer: } The Lindel\"of property is preserved under surjections. See \ref{MunkresP30.11}.
    \item How does being Lindel\"of behave under quotients?

    \textbf{Answer:} Since the Lindel\"of property is surjective, it is also preserved under quotients.
    \item How does (topological) separability behave under taking subspaces?
    
    \textbf{Answer:} Separability is not preserved under taking subspaces, not even closed ones. This is shown by the antidiagonal subspace \( \Delta^- \) of the Sorgenfrey line. See \hyperlink{MunkresP30.9}{the relevant Munkres problem}.
    \item How does separability behave under products?
    \item How does separability behave under coproducts?
    \item How does separability behave under surjections?

    \textbf{Answer:} Separability is preserved under surjections, see \hyperlink{MunkresP30.11}{this Munkres problem}.
    \item How does separability behave under quotients?
    
    \textbf{Answer:} Since separability is surjective, it is preserved under quotients.
\end{enumerate}

\section{The separation axioms}

\subsection{Definitions}

\begin{defn}[Hausdorff vel \( T_2 \)]
\label{Hausdorff}
\hypertarget{Hausdorff}
A topological space \( X \) is called \emph{Hausdorff} or \( T_2 \) iff for every two points \( x_1, x_2 \in X \) there are disjoint open neighbourhoods containing them.
\end{defn}

\begin{defn}[Regularity vel \( T_3 \)]
\label{RegularSpace}
\hypertarget{RegularSpace}
A topological space \( X \) is called \emph{regular} or \( T_3 \) iff every singleton is closed and for any point \( x \in X \) and closed subset \( F \subseteq X \) there are disjoint open neighbourhoods \( U \ni x, V \supseteq F \).
\end{defn}

\begin{defn}[Normality vel \( T_3 \)]
\label{NormalSpace}
\hypertarget{NormalSpace}
A topological space \( X \) is called \emph{normal} or \( T_4 \) iff every singleton is closed and for any two closed subsets \( F, G \subseteq X \) there are disjoint open neighbourhoods \( U \supset F, V \supseteq G \).
\end{defn}
\subsection{Lemmas}

\begin{lemma}[Regularity via shrinking]
\label{RegularityShrinking}
\hypertarget{RegularityShrinking}
Let \( X \) be a space with closed singletons. Then \( X \) is \( T_3 \) iff for every point \( x \) and its neighbourhood \( U \ni x \) we can find a smaller neighbourhood \( V \ni x \) such that \( \Cl V \subseteq U \).
\end{lemma}

\begin{proof}
    In the forward direction, \( U^c \) is a closed set disjoint from \( x \), to there by the \( T_3 \) axiom there are open neighbourhoods \( V \in x \) and \( W \supseteq U^c \). In particular, every point of \( U^c \) is separated from \( V \) by the neighbourhood \( W \), so \( \Cl V \cap U^c = \varnothing \) or, equivalently, \( \Cl V \subseteq U \).

    In the backward direction, for every closed set \( F \not\ni x \), \( F^c \) is a neighbourhood of \( x \). If we shrink it to a \( V \) with \( \Cl V \subseteq F^c \), we get an open neighbourhood \( (\Cl V)^c \supseteq F \) disjoint with V.
\end{proof}

\begin{lemma}[Normality via shrinking]
\label{NormalityShrinking}
\hypertarget{NormalityShrinking}
Let \( X \) be a space with closed singletons. Then \( X \) is \( T_3 \) iff for every closed subset \( F \subset X \) and its neighbourhood \( U \supset F \) we can find a smaller neighbourhood \( V \supset F \) such that \( \Cl V \subseteq U \).
\end{lemma}

\begin{proof}
In the forward direction, \( U^c \) is a closed set disjoint from \( A \), to there by the \( T_4 \) axiom there are open neighbourhoods \( V \supseteq F \) and \( W \supseteq U^c \). In particular, every point of \( U^c \) is separated from \( V \) by the neighbourhood \( W \), so \( \Cl V \cap U^c = \varnothing \) or, equivalently, \( \Cl V \subseteq U \).

In the backward direction, for every two closed sets \( G \cap F = \varnothing \), \( G^c \) is a neighbourhood of \( F \). If we shrink it to a \( V \) with \( \Cl V \subseteq G^c \), we get an open neighbourhood \( (\Cl V)^c \supseteq G \) disjoint with V.
\end{proof}
\subsection{Theorems}

\subsection{Problems}

\subsection{Chapter review questions}

\begin{enumerate}
    \item Why do the axioms \( T_3 \) and \( T_4 \) have assumptions about closed singletons?

    \item What is the shrinking neighbourhood characterisation of regularity?

    \item What is the shrinking neighbourhood characterisation of normality?

    \item What are the relations between the axioms \( T_2, T_3, T_4 \)?

    \item What separation axioms does the Sorgenfrey line satisfy?

    \item What is the significance of the Sorgenfrey plane for separation axioms? Which separations axioms does is satisfy?

    \item What is the significance of the space \( \mathbb{R}_K \) for separation axioms? Which separation axioms does is satisfy?
\end{enumerate}

\subsection{Questions about property preservation}

\begin{enumerate}
    \item How does the \( T_2 \) axiom behave under taking subspaces?
    \item How does the \( T_2 \) axiom behave under products?
    \item How does the \( T_2 \) axiom behave under coproducts?
    \item How does the \( T_2 \) axiom behave under surjections?
    \item How does the \( T_2 \) axiom behave under quotients?
    \item How does the \( T_3 \) axiom behave under taking subspaces?
    \item How does the \( T_3 \) axiom behave under products?
    \item How does the \( T_3 \) axiom behave under coproducts?
    \item How does the \( T_3 \) axiom behave under surjections?
    \item How does the \( T_3 \) axiom behave under quotients?
    \item How does the \( T_4 \) axiom behave under taking subspaces?
    \item How does the \( T_4 \) axiom behave under products?
    \item How does the \( T_4 \) axiom behave under coproducts?
    \item How does the \( T_4 \) axiom behave under surjections?
    \item How does the \( T_4 \) axiom behave under quotients?
\end{enumerate}
