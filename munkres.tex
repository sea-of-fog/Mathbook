\chapter{Munkres}

\section{The Countability Axioms}

\subsection{Definitions}

\begin{defn}[Basis at a point]
\label{BasisAtPoint}
\hypertarget{BasisAtPoint}
Let \( x \in X \). A collection \( \mathcal{B} \) of neighbourhoods of \( x \) is a \textbf{basis at \( x \)} iff every open neighbourhood \( U \ni x \) contains a member of \( \mathcal{B} \), i.e. there is a \( B \in \mathcal{B} \) s. that \( x \in B \subseteq U \).
\end{defn}

\begin{defn}[First countability]
\label{FirstCountable}
\hypertarget{FirstCountable}
A space \( X \) is said to be \textbf{first-countable} if it has a countable basis \( \mathcal{B}_x \) at each of its points \( x \in X \).
\end{defn}

\begin{defn}[Second countability]
\label{SecondCountable}
\hypertarget{SecondCountable}
A space \( X \) is said to be \textbf{second-countable} if it has a countable basis.
\end{defn}

\begin{defn}[Dense subset]
\label{DenseSubset}
\hypertarget{DenseSubset}
A subset \( D \subseteq X \) is said to be \textbf{dense} if
\[ 
    \Cl D = X. 
\]
\end{defn}

\begin{defn}[Lindel\"of property]
\label{Lindelof}
\hypertarget{Lindelof}
A space \( X \) is called \textbf{Lindel\"of} if every open cover of \( X \) admits a countable subcover.
\end{defn}

\subsection{Theorems}

\begin{thm}[Continuity in first-countable spaces]
\label{FirstCountableContinuousFunctions}
\hypertarget{FirstCountableContinuousFunctions}
If \( f: X \to Y \) is continuous, then for all sequences \( x_n \to x \) we have \( f(x_n) \to f(x) \). If additionally \( X \) is first-countable, then then converse is also true. 
\end{thm}

\begin{proof} 
    For the forward implication, pick any \( U \in N_{f(x)} \). Then \( V := f^{-1}[U] \) is open, so, for all but finitely many indices \( n \), \( x_n \in V \). Then for the same indices \( f(x_n) \in U \).

    For the backward, let \( B_n \) be a countable basis of X at \( x \) and suppose without loss of generality that \( B_n \) is decreasing. Pick again an open neighbourhood \( U \in N_{f(x)} \). If \( f^{-1}[U] \) is not open, then choose a sequence \( x_n \in B_n \setminus f^{-1}[U] \). This means that \( f(x_n) \not\in U \) for all \( n \), but \( f(x_n) \to f(x) \), a contradiction.
\end{proof}

\subsection{Problems}

\begin{problem}[P30.18 from \cite{Mun14}]
\label{MunkresP30.18}
\hypertarget{MunkresP30.18}
    Let \( G \) be a first-countable topological group. Show that if \( G \) is either separable or Lindel\"of, then \( G \) is also second-countable.
\end{problem}

\begin{proof}[Solution (for separable spaces)]
    Let \( D = \left\{ d_1, d_2, \ldots \right\} \) be the countable dense subset and \( \mathcal{B} = \left\{ B_1, B_2 \ldots \right\} \) a basis at \( e \). We will use the dense subset to \emph{spread} the local basis over the group. Concretely, we claim that the countable basis can be given by
    \[ 
       d_m B_n,
   \]
   i.e. left translations of the basis at \( e \). Pick any open set \( U \subseteq G \) and its element \( g \in U \subseteq G \). The left-transalted set \( g^{-1}U \) contains \( e = g^{-1}g \), so we can pick a neighbourhood \( B_n \subseteq g^{-1}U \) or, equivalently, \( g \in gB_n \subseteq U \).
   At this point it might be true that for some \( d_m \) we will have \( g \in d_mB_n \subseteq U \). However, by continuity of the group operations we can shrink \( B_n \) to \( B_k \) such that \( B_k \cdot B_k \subseteq B_m \). Then, pick a \( d_m \in gB_k \). We can now compute that
\begin{align*}
    d_m &\in gB_k  \\
    g^{-1}d_m &\in B_k \\
    e \in g^{-1}d_mB_k &\subseteq B_kB_k \subseteq B_m \\
    g \in d_mB_k &\subseteq gB_m \subseteq U, \\
\end{align*}
which completes the proof.
\end{proof}

\begin{proof}[Solution (for Lindel\"of spaces)]
    Let \( \mathcal{B} = \left\{ B_1, B_2, \ldots \right\} \) be a countable basis at \( e \). For a given \( B_k \), consider the cover
    \[ 
       \mathcal{A_k} = \left\{ gB_k : g \in G \right\}.
   \]
   By the Lindel\"of property, we can pick a countable subset \( G_k \subseteq G \) such that the family \( \left\{ gB_k : g \in G_k \right\} \)is a subcover of \( \mathcal{A_k} \). Now do this for all \( k \in \mathbb{N} \) and, for convenince, take
   \[ 
       %TODO: upward sum
       G_0 := \bigcup_{k=1}^\infty G_k. 
  \]
  We claim that sets of the form
  \[ 
     g_0B_k 
 \]
 for \( g_0 \in G_0 \) and \( B_k \in \mathcal{B} \) form a countable basis. Pick any open set and its element \( g \in U \subseteq G \). We repeat the solution of the previous subproblem. To be explicit, we can find an \( m \) such that \( gB_m \subseteq U \). Now pick a \( B_k \) with the property that \( B_k \cdot B_k \subseteq B_m \). Since the family \( \left\{ gB_k : g \in G_0 \right\} \), we can find some \( g_0 \) s. that \( g \in g_0B_k \subseteq gB_m \subseteq U \).
\end{proof}

\paragraph{Remark.} The set \( G_0 \) is actually a dense set for the topology of \( G \).

\paragraph{Remark.} This is one of the situations in which the relationships between countability axioms can be reversed, that is a subset of weaker axioms implies second-countability.

\subsection{Chapter review questions}

\begin{enumerate}
    \item What is the significance of first-countability for closures?
    \item What is the significance of first-countability for continuous function?

    \textbf{Answer:} If \( X \) is first-countable, then a function \( f: X \to Y \) is continuous iff for all sequences \( x_n \to x \) we have \( f(x_n) \to f(x) \). In general, only the forward implication holds. Prove \hyperlink{FirstCountableContinuousFunctions}{this}.
    
    \item What examples does Munkres give for a first-countable space that is not second-countable?

        \textbf{Answer:} \( \mathbb{R}^\omega \) is first-countable as it is a metric space, but is not second-countable, because it has a discrete set \( \{ 0,\,1 \}^\omega \) of size \( \mathfrak{c} \).
    \item How large can a discrete subspace of a second-countable space be?
    
    \textbf{Answer:} At most countable. A discrete subspace can be injected into the basis, so it has a smaller size.
    \item What are the relations between countability propeties?

    \textbf{Answer:} Second-countability implies all other properties. See \hyperlink{MetrisableSpaceCountabilityProperties}{here} for a proof. Not even a conjuction of the three others implies second-countability, as shown by the Sorgenfrey line \( \mathbb{S} \).
    \item What is the significance of the Sorgenfrey line \( \mathbb{S} \) for countability axioms? Which countability properties does it have?
    \item What is the significance of the Sorgenfrey plane \( \mathbb{S}^2 \) for countability axioms? Which countability properties does it have?

        \textbf{Answer:}  The Sorgenfrey plane is first-countable, separable, but neither second-countable nor Lindel\"of. Thus it is a product of Lindel\"of spaces which is not Lindel\"of, establishing that Lindel\"ofness is not productive. The proof uses the fact that the antidiagonal is a discrete subspace.

    \item What is the significance of the ordered square \( I_o^2 \) for countability axioms? Which countability properties does it have?
    \item How can you make it easier to check the Lindel\"of property?
    \item Under what conditions do the relations between countability axioms reverse, i.e. what can you use to conclude that a space is second-countable from the other axioms?
    
        \textbf{Answer:} (1) metrisable spaces are all first-countable, but either the Lindel\"of property or separability imply second-countability (\hyperlink{MetrisableSpaceCountabilityProperties}{proof}) (2) a first-countable topological group satisfying either the Lindel\"of property or separability is second-countable (\hyperlink{MunkresP30.18}{proof}).
\end{enumerate}

\subsection{Questions about property preservation}
\begin{enumerate}
    \item Is first-countability hereditary?
    \item Is first-countability a productive property?
    \item Is first-countability a coproductive property?
    \item Is first-countability a surjective propety?
    \item is first-countability preserved under quotients?
    \item Is second-countability hereditary?
    \item Is second-countability a productive property?
    \item Is second-countability a coproductive property?
    \item Is second-countability a surjective propety?
    \item is second-countability preserved under quotients?
    \item Is being Lindel\"of hereditary?
    \item Is being Lindel\"of a productive property?
    \item Is being Lindel\"of a coproductive property?
    \item Is being Lindel\"of a surjective propety?
    \item is being Lindel\"of preserved under quotients?
    \item Is separability hereditary?
    \item Is separability a productive property?
    \item Is separability a coproductive property?
    \item Is separability a surjective propety?
    \item Is separability preserved under quotients?
\end{enumerate}
