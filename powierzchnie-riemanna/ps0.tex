\chapter{Powierzchnie Riemanna, Problemset 0}

% TODO: tw. Liouville'a
% TODO: rodzaje osobliwości

\section*{Zadanie 2}

Obicnamy \( f \) do \( \mathbb{C} \subseteq \overline{ \mathbb{C} } \). Ponieważ \( \overline{ \mathbb{C} } \) jest zwarta, to \( f \) obcięta do \( \mathbb{C} \) jest ograniczona, więc z \textbf{Tw. Liouville'a} \( f \) jest stała. Niestałą funkcją \( g: \overline{ \mathbb{C} } \to \overline{ \mathbb{C} } \) jest identyczność.

\section*{Zadanie 6}

Definiujemy
\[ 
    \varphi_w(z) = \frac{z - w}{1 - \overline{w}z}.
\]


\paragraph{Uwaga.} Łatwiej to rozwiązać po sprowadzeniu rotacją \( w \) na dodatnią półoś rzeczywistą.

\section*{Zadanie 8}

\begin{enumerate}
    \item
    % TODO: przypadek c = 0
    % TODO: sprawdzenie
    Określamy \( \psi_A(-d/c) = \infty \) oraz \( \psi_A(\infty) = a/c \). 

    \item
\end{enumerate}

\section*{Zadanie 9}

Licznik otrzymanej homografii \( \psi_A(x) \) musi być krotnością \( x - z \), ponieważ \( \psi_{cA} = \psi{A} \), to b.s.o.
\[ 
    A =  
    \begin{pmatrix}
        1 & -z \\
        \alpha & \beta
    \end{pmatrix}.
\]
Z równań \( A.u = 1 \) oraz \( A.w = \infty \) otrzymujemy
\begin{align}
    u - z &= \alpha u + \beta \\
    \alpha w + \beta &= 0, \\
\end{align}
skąd wyliczamy
\begin{align}
    \alpha &= \frac{u-z}{u - w} \\ 
    \beta &= -w\frac{u-z}{u-w}.
\end{align}

\section*{Zadanie 10}

\begin{enumerate}
    \item Można skorzystać z poprzedniego zadania dla \( f(0), f(1), f(\infty) \).

    \item Rozszerzamy \( h(z) = g(z)/z \) o \( h(0) = g'(0) \) oraz \( h(\infty) = 1 \). Wtedy \( h: \overline{\mathbb{C}} \to \mathbb{C} \) jest ciągłą funkcją, jest więc ograniczona. Holomorficzność na \( \mathbb{C} \) daje \textbf{TODO: twierdzenie o holomorficznej dziurze}.
\end{enumerate}

\section*{Zadanie 11}

Niech \( g: \mathbb{C} \to \mathbb{C} \) będzie biholomorfizmem. Określamy \( \overline{g}: \overline{ \mathbb{C} } \to \overline{ \mathbb{C} } \), rozszerzając \( g \) i kładąc \( \overline{g}(\infty) = \infty \).
