\chapter{Powierzchnie Riemanna, Problemset 0}

% TODO: tw. Liouville'a
% TODO: rodzaje osobliwości

\section*{Zadanie 2}

Obicnamy \( f \) do \( \mathbb{C} \subseteq \overline{ \mathbb{C} } \). Ponieważ \( \overline{ \mathbb{C} } \) jest zwarta, to \( f \) obcięta do \( \mathbb{C} \) jest ograniczona, więc z \textbf{Tw. Liouville'a} \( f \) jest stała na \( \mathbb{C} \), a więc z ciągłości także na.

Niestałą funkcją \( g: \overline{ \mathbb{C} } \to \overline{ \mathbb{C} } \) jest identyczność, także każdy wielomian dodatniego stopnia i niestała funkcja wymierna.

\section*{Zadanie 4}

\subsection*{Sposób brzydki (mój)}

Wyrzucamy z dziedziny \( \infty \) oraz \( f^{-1}(\infty) \), otrzymująć nakłute \( \mathbb{C} \), a więc otwarty region w płaszczyźnie zespolonej (po obu stronach!). Do nich stosujemy tw. o odwzorowaniu odwrotnym, natomiast pozostałe punkty są osobliwościami usuwalnymi.

\subsection*{Sposób ładny (G.C.)}

Korzystamy z jednego z następnych zadań by zapewnić sobie, że \( f(\infty) = \infty \). Wtedy \( f|_\mathbb{C} \) jest bihomolorfizmem \( \mathbb{C} \) z definicji funkcji holomorficznej na sferze Riemanna. Żeby zbadać holomorficzność w \( \infty \) odnotujemy, że \( 1/f(1/z) \) ma w \( 0 \) osobliwość usuwalną, bo \( f \) jest ciągłe na \( \overline{\mathbb{C}} \).

\section*{Zadanie 6}

Definiujemy
\[ 
    \varphi_w(z) = \frac{z - w}{1 - \overline{w}z}.
\]

% Patrz: Fornaess

\paragraph{Uwaga.} Łatwiej to rozwiązać po sprowadzeniu rotacją \( w \) na dodatnią półoś rzeczywistą. Dokładniej: podkładamy funkcję \( z \mapsto \frac{\overline{w}}{ \left| w \right|  }z \). Wtedy, żeby udowodnić, że obraz dysku zawiera się w dysku, potrzebna jest nierówność geometryczna: \textit{naprzeciwko większego kąta w trapezie jest większa przekątna}.

\paragraph{Motywacja.} Ten wzór można znaleźć, szukając inwersji, która zachowa okrąg jednostkowy (a więc będzie względem okręgu ortogonalnego do niego!) oraz zamieni miejscami \( 0 \) i \( w \).

\section*{Zadanie 7}

Korzystając z poprzedniego zadania możemy rozważyć \( g = \varphi_{f(0)} \circ f \), zatem \( g(0) = 0 \). Wtedy \( g'(0) \cdot (g^{-1})'(0) = 1 \), ale skoro obie te liczby są z lematu Schwarza co do modułu nie większe niż \( 1 \), to obie są co do modułu równe 1, więc \( g(z) = az \), czyli \( f = \varphi_{-g(0)} \circ (z \mapsto az) \), gdzie \( a = f'(0) \) i \( |a| = 1 \). Zatem wzór można napisać
\[ 
    \frac{\omega z - w}{1 - \overline{w}\omega z}.
\]

\section*{Zadanie 8}

\paragraph{Sposób gorszy.} Można na palcach zdefiniować \( \psi_A(-d/c) = \infty \) oraz \( \psi_A(\infty) = a/c \), formalnie rozważając osobno przypadek \( c = 0 \) (wtedy \( \psi_A \) jest liniową funkcją na \( \mathbb{C} \) oraz trzyma \( \infty \)).

\subsection*{(a)}
Na palcach, jak w \textit{sposobie gorszym}, definiujemy \( \psi_A \). Przypadek \( c = 0 \) jest trywialny. W pozostałych przypadkach mamy do sprawdzenia trzy pary map.
\[ 
    \psi_A(1/z) = \frac{a/z + b}{c/z + d} = \frac{bz + a}{dz + c} 
\]
\[ 
    1/\psi_A(1/Z) =  \frac{dz + c}{bz + a}
\]

\subsection*{(b)}

Korzystamy z interpretacji \( \mathbb{C} \) jako \( \mathbb{C}P^1 \). Wtedy to działanie jest zrzuceniem do ilorazu naturalnego działania \( \mathrm{GL}(2, \mathbb{C}) \) na \( \mathbb{C}^2 \setminus \left\{ 0 \right\} \). To, że do ilorazu można przejść wynika z jednorodności funkcji liniowych.


\section*{Zadanie 9}

Licznik otrzymanej homografii \( \psi_A(x) \) musi być krotnością \( x - z \), ponieważ \( \psi_{cA} = \psi{A} \), to b.s.o.
\[ 
    A =  
    \begin{pmatrix}
        1 & -z \\
        \alpha & \beta
    \end{pmatrix}.
\]
Z równań \( A.u = 1 \) oraz \( A.w = \infty \) otrzymujemy
\begin{align*}
    u - z &= \alpha u + \beta \\
    \alpha w + \beta &= 0, \\
\end{align*}
skąd wyliczamy
\begin{align*}
    \alpha &= \frac{u-z}{u - w} \\ 
    \beta &= -w\frac{u-z}{u-w}.
\end{align*}

\section*{Zadanie 10}

\begin{enumerate}
    \item Można skorzystać z poprzedniego zadania dla \( f(0), f(1), f(\infty) \).

    \item Rozszerzamy \( h(z) = g(z)/z \) o \( h(0) = g'(0) \) oraz \( h(\infty) = 1 \). Wtedy \( h: \overline{\mathbb{C}} \to \mathbb{C} \) jest ciągłą funkcją, jest więc ograniczona.

        Sprawdzamy w mapie. Wiadomo, że \( 1/g(1/z) \) jest holomorficzne w otoczeniu zera i ciągłe w zerze, więc jest holomorficzne i ma tam wartość zero, więc \( 1/g(1/z) = zk(z) \) przy \( k(0) \neq 0 \) (funkcja musi być 1-1, więc ma niezerową pochodną w zerze). Holomorficzność na \( \mathbb{C} \) daje \textbf{TODO: twierdzenie o holomorficznej dziurze}, że \( h \) jest holomorficzne na \( \mathbb{C} \), więc \( h \) jest stałe z zad. 2. (albo z Tw. Liouville'a).

    \item Skoro \( h(z) = a \), to \( g(z) = zh(z) = az \), więc \( f = \psi_{A^{-1}} \circ g = \psi_{A^{-1}}  \) z jednorodności działania \( \mathrm{GL}(2, \mathbb{C}) \) na \( \overline{\mathbb{C}} \).
\end{enumerate}

\section*{Zadanie 11}

Niech \( g: \mathbb{C} \to \mathbb{C} \) będzie biholomorfizmem. Określamy \( \overline{g}: \overline{ \mathbb{C} } \to \overline{ \mathbb{C} } \), rozszerzając \( g \) i kładąc \( \overline{g}(\infty) = \infty \). Holomorficzność tej funkcji opiera się na następującym lemacie.

\begin{lemma}[Holomorficzna funkcja wybucha w nieskończoności]
\label{BiholomorphismExplodesAtInfinity}
\hypertarget{BiholomorphismExplodesAtInfinity}
Niech \( f: \mathbb{C} \to \mathbb{C} \) będzie biholomorfizmem. Wtedy
\[ 
    \lim\limits_{z \to \infty} f(z) = \infty.
\]
\end{lemma}

\begin{proof}[Dowód (lematu)]
    Załóżmy nie wprost, że teza nie zachodzi. Wtedy istnieje \( R > 0 \) i ciąg \( z_n \) uciekający do nieskończoności (\( \left| z_k \right| \to \infty  \)) taki, że \( \left| f(z_n) \right| < R \) dla wszystkich \( n \). Ponieważ \( \overline{B}(0, R) \) jest zwartym zbiorem, ciąg \( f(z_n) \) ma punkt skupienia. Ale wtedy także \( z_n \) ma punkt skupienia, bo \( f \) ma inwers, co daje sprzeczność (bo \( z_n \) ucieka do \( \infty \)).
\end{proof}

Sprawdzenie holomorficzności na \( \overline{g}^{-1}[ \mathbb{C} ] \) jest trywialne (tam \( \overline{g} = g \)). W otoczeniu \( \overline{g}^{-1}(\infty) = \infty \) sprawdzamy \textit{zwykłą} holomorficzność funkcji \( 1/g(1/z) \) wokół \( 1/\infty = 0 \). Z lematu wyżej wynika, że jest ona tam ciągła, więc ma osobliwość usuwalną.

Funkcja \( g \) jest zatem homografią. Jedyne homografie, które są funkcjami całkowitymi, to funkcje liniowe, co kończy dowód.
